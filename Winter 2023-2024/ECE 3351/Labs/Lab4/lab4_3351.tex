\documentclass[]{report}
\usepackage{parskip}
\usepackage[]{fbb}
\usepackage[top=45mm, bottom=45mm, left=25mm, right=25mm]{geometry}
\usepackage{listings}
\usepackage{tikz}
\usepackage{amsmath}
\usetikzlibrary{positioning, shapes.geometric, arrows}
\usepackage{mathrsfs}

\title{\textbf{Systems Described by Linear Constant-Coefficient Difference Equations}}
\date{\textit{\today}}
\author{Arnav Goyal - 251244778}

\newcommand{\ztrans}[1]{Z\left\{ #1 \right\}}

\begin{document}
\maketitle
\section*{The Difference Equation}

This laboratory report constantly refers to the difference equation in (1)
\begin{equation}
	y(n) = y(n-1) - 0.8y(n-2) + x(n)
\end{equation}

We can represent (1) as a $z$-plane transfer function through the following steps:
	\[ \ztrans{y(n)} = \ztrans{ y(n-1) - 0.8y(n-2) + x(n) } \]
	\[ Y = z^{-1}Y -0.8z^{-2}Y + X	\]
	\[ Y - z^{-1}Y + 0.8z^{-2}Y = X	\]
	\[ Y\left( 1 - z^{-1} + 0.8z^{-2}\right) = X\]
	\[\frac{Y}{X} = \frac{1}{ 1 - z^{-1} + 0.8z^{-2}}\]

Essentially we can represent this difference equation as a transfer function with numerator coefficients $b = \left[	1\right]$, and denominator coefficients $a = \left[1, -1, 0.8\right]$. This is what is implemented at the top of every \texttt{MATLAB} script.

\section*{Impulse Response}

In order to calculate and plot the Impulse Response, the following \texttt{MATLAB} script was created.
\vspace{1em}

\begin{lstlisting}[frame=single, language=Matlab]
%% QUESTION 1
clear;

% define the difference equation
b = 1; % numerator (output) x coeffs
a = [1, -1, 0.8]; % denominator (input) y coeffs

% finding impulse resp
t = -20:200;
[h, n] = impz(b, a, length(t));

% plotting impulse resp
stem(n, h);
grid on;
title('Impulse Response');
xlabel('n');
ylabel('h(n)');
\end{lstlisting}

This script produced the following output plot for the impulse response
\vspace{1em}

\begin{center}
	\includegraphics[scale=0.35]{Impulse Resp}
\end{center}

\section*{Step Response}

In order to calculate and plot the Step Response the following \texttt{MATLAB} script was created.
\vspace{1em}

\begin{lstlisting}[frame=single, language=Matlab]
%% QUESTION 2
clear;

% define the difference equation
b = 1; % numerator (output) x coeffs
a = [1, -1, 0.8]; % denominator (input) y coeffs

% find step resp
t = -20:200; % time scale (x-axis)
u = ones(size(t));
[s, final] = filter(b, a, u); 
% filters data in 't' according to num 'b' and den 'a', outputs this in 'y'

% plot step resp
stem(t,s);
grid on;
title('Step Response');
ylabel('s(n)');
xlabel('n');

\end{lstlisting}

This script produced the following output plot for the step response
\vspace{1em}

\begin{center}
	\includegraphics[scale=0.35]{Step Resp}
\end{center}

\section*{System Stability}

In order to test system stability, we must check if sum of the absolute values of the impulse response converges.
In other words we must establish that (2) holds true:
\begin{equation}
	\sum_{n=0}^{\infty} \left| h(n) \right| < \infty
\end{equation}
To test that, the following \texttt{MATLAB} script was created.
\vspace{1em}

\begin{lstlisting}[frame=single, language=Matlab]
%% QUESTION 3

% define the difference equation
b = 1; % numerator (output) x coeffs
a = [1, -1, 0.8]; % denominator (input) y coeffs

% finding impulse resp
t = -20:200;
[h, n] = impz(b, a, length(t));

% check stability criterion, does the summation converge?
val = sum(abs(h));
stable = val < inf; % this is a boolean 1=True, 0=False
disp(['System Stable? : ' num2str(stable)]);
disp(['Converges to : ' num2str(val) ' on -20:200']);
\end{lstlisting}

This script produced the following output for the system stability. 

\textit{Note:} Within this script a '1' means Stable, and a '0' means unstable.

\begin{center}
	\includegraphics[scale=0.75]{Stability Resp}
\end{center}

\end{document}