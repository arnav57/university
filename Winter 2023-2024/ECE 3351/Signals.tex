\chapter{Introduction to Signal Processing}

\section{Signals \& Systems}

The study of signal processing involves \keyword{signals}\sidenote{A signal is a measurable or detectable physical quantity (air-pressure, voltage, current, etc.) that carries some information} and \keyword{systems}\sidenote{A system is an object that performs an operation on a signal}.
\begin{bullets}
	\item Real-life signals contain \keyword{information}, and \keyword{noise}\sidenote{noise is just the part of the signal we are not interested in}
	\item Information can be encoded within a signal through modulating its amplitude, frequency, phase or spectral content.
\end{bullets}

Signal processing is basically the study of extracting the encoded information within an \keyword{analog} or \keyword{digital} signal.
\begin{bullets}
	\item Analog signals vary continuously with respect to their independent variables
	\item Digital signals are discrete-time and discrete-valued signals
\end{bullets}

Digital Signals have some amazing benefits\sidenote{Noise robustness, software implementation, ease of storage and transmitting, and independent of external factors like temperature}, and drawbacks\sidenote{Increased system complexity due to A/D and D/A conversion, Limited Frequency Range, Increased power consumption - requires active devices} too. Overall the benefits outweight the drawbacks.

\section{Signal Classification}
Signals possess several characteristics that allow us to classify them further, below is a table of the characteristics and a short description of each.

\begin{fullwidth}
\begin{table}[h]
	\begin{tabular}{@{}ll@{}}
		\toprule
		\multicolumn{1}{c}{Signal Category} & \multicolumn{1}{c}{Description} \\ \midrule
		Discrete Time                                               & Defined on a set of numbers - a sequence                \\
		Continuous Time                                             & Defined on all real numbers - a function                \\ \midrule
		Discrete Valued                                             & Take on values from a finite set                        \\
		Continuous Valued                                           & Takes on a continuous range of values                   \\ \midrule
		Single Channel                                              & From One Source, Represented by a scalar function       \\
		Multi-Channel                                               & From Many Sources, Represented by a vector function     \\ \midrule
		Single Dimension                                            & Functions of one independent variable (usually time)    \\
		Multi-Dimension                                             & Functions of more than one independent variable         \\ \midrule
		Deterministic                                               & Signal is exactly predictable with past values only     \\
		Random                                                      & Signal is NOT exactly predictable with past values only \\ \bottomrule
	\end{tabular}
\end{table}
\end{fullwidth}
\vspace{2em}
Consider the picture of a color TV. The signal would be a three channel, three dimensional signal. Which we could represent mathematically as below.

\begin{displaymath}
	\mathbb{I}\left( x,y,t \right) = \begin{bmatrix}
		I_r\left(x,y,t\right) \\ I_g\left(x,y,t\right) \\ I_b\left(x,y,t\right)
	\end{bmatrix}
\end{displaymath}