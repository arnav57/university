\chapter{Introduction}

\section{Signals \& Systems}

As we begin the study of \keyword{signal processing} we should start by defining what a \keyword{signal} actually is. 

\begin{definition}
	A \keyword{signal} is a measurable or detectable physical quantity (air-pressure, voltage, current, etc.) that carries some information.
\end{definition}

The study of signal processing involves passing a signal through a \keyword{system}. With the goal of extracting something (usually the information) from the signal
\begin{definition}
	A \keyword{system} is a physical device that performs an operation on a signal.
\end{definition}

\section{Information \& Noise}
Signals will always contain two things. \keyword{Information} and \keyword{Noise}. The information is the part we are interested in, and the Noise is considered everything else. For example when dealing with speech recognition, background music would be considered noise.

Information can be encoded within a signal through the some of the following methods:
\begin{itemize}
	\item Signal Amplitude
	\item Signal frequency or Spectral Content
	\item Signal Phase
\end{itemize}

Now we can see that signal processing deals with extracting information by essentially 'interpreting' its used encoding method.

\subsection*{Analog \& Digital Signals}

Signals can be found as \keyword{analog signals} or as \keyword{digital signals}.
\begin{definition}
	An \keyword{analog signal} is a signal that varies continuously with respect to its independent variables.
\end{definition}

Most real life signals are analog signals, and analog signals can either be processed directly or converted into digital signals for processing before being reconverted into analog signals.

\begin{definition}
	A \keyword{digital signal} is a signal that represents data as a sequence of discrete and quantized values. In other words, it is a discrete-valued AND discrete-time signal.
\end{definition}

Digital signals have some pretty big benefits:
\begin{itemize}
	\item Noise robustness - either a 0 or 1
	\item Software Implementation - can be processed through code instead of circuits
	\item Easy to store and transmit
	\item Fairly independent of external parameters like temperature
\end{itemize}

They have some drawbacks too:
\begin{itemize}
	\item Increased system complexity - Need A/D and D/A conversion and complex digital circuitry
	\item Limited Range of Frequencies - Sampling rate must be twice the value of the highest frequency present within the signal
	\item Power Consumption - requires active devices, rather than analog signal processors which can be passive.
\end{itemize}

Overall, the advantages matter more than the drawbacks which is why digital signals are so commonplace.

\section{Classifying Signals}

Signals can be further classified based on the information they provide. Namely to classify a signal we choose an option among the 5 characteristics below.
\begin{itemize}
	\item Continuous-Time OR Discrete-Time
	\item Continuous-Valued OR Discrete-Valued
	\item One-Channel OR Multi-Channel
	\item One-Dimensional OR Multi-Dimensional
	\item Deterministic OR Random
\end{itemize}

\subsection*{Continuous \& Discrete-Time Signals}

\keyword{Continuous-time signals} are defined on some interval $(a,b)$, thus it can be represented as a function of a continuous variable $f\left( t \right) , t\in\left[ a,b\right]$

\keyword{Discrete-time signals} are defined on a set of integer numbers (indices) $\left\{ 0,1,2,\ldots \right\}$ and each index has a corresponding value $\left\{ f_0,f_1,f_2,\ldots \right\}$. This means that we can represent them as sequences: $f\left[ n\right] = \left\{ f_0,f_1,f_2,\ldots \right\}$

\subsection*{Continuous \& Discrete-Valued Signals}

\keyword{Continuous-valued signals} can take on all-possible values from its continuous range, while
\keyword{discrete-valued signals} take on values from a finite set. The process of converting a continuous-valued signal into a discrete-valued one is called \keyword{quantization}

\subsection*{Signal Channels}
\keyword{single channel signals} are usually generated by one source and can be represented by \emph{scalar} functions, \keyword{multi-channel signals} are usually generated by multiple sources/sensors and are represented by \emph{vector functions}

\subsection*{Signal Dimensions}

\keyword{One-dimensional signals} are a function of one independent variable (usually time). while \keyword{N-dimensional signals} are a function of N independent variables (time, x-pos, y-pos, etc.)

\subsection*{Deterministic \& Random Signals}

If we can exactly predict the future values of a signal by using its past values, we say that the signal is a \keyword{deterministic signal}. On the other hand, if we cannot predict it exactly we say that the signal is a \keyword{random signal}. There is no sharp distinction between the two due to the presence of noise, but these serve the purpose of broad categories.

Consider

\begin{displaymath}
	\mathbb{I}\left( x,y,t \right) = \begin{bmatrix}
	I_r\left(x,y,t\right) \\ I_g\left(x,y,t\right) \\ I_b\left(x,y,t\right)
	\end{bmatrix}
\end{displaymath}

	 
