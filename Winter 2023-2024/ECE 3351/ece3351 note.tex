% BEGIN_FOLD PREAMBLE
\documentclass[]{report}

%BEGIN_FOLD IMPORTS
\usepackage{amsfonts}
\usepackage[]{tikz}
\usetikzlibrary{shapes.geometric}
\usepackage[]{xcolor}
\usepackage[]{anyfontsize}
\usepackage[top=20mm, bottom=20mm, left=25mm, right=25mm]{geometry}
\usepackage[explicit]{titlesec}
\usepackage{fancyhdr}
\usepackage[most]{tcolorbox}
\usepackage{listings}
\usepackage[]{parskip}
\usepackage{caption}
\usepackage{enumitem}
\usepackage[hidelinks]{hyperref}
\hypersetup{linktoc=all}
\usepackage[outline]{contour}
\usepackage{soul}
\sethlcolor{greyblue!95!white}
% END_FOLD IMPORTS

% BEGIN_FOLD COLORS
\definecolor{whitesmoke}{HTML}{CCCCCC}
\definecolor{grey}{HTML}{686E75}
\definecolor{red}{HTML}{D65B5C}
\definecolor{orange}{HTML}{EF8511}
\definecolor{yellow}{HTML}{D3B006}
\definecolor{green}{HTML}{5DB267}
\definecolor{blue}{HTML}{62ABF3}
\definecolor{purple}{HTML}{7F76DA}
\definecolor{pink}{HTML}{B76CC7}
\definecolor{greyblue}{HTML}{171B20} % default: 171B20

\pagecolor{greyblue}
\AtBeginDocument{\color{whitesmoke}}


% END_FOLD COLORS

% BEGIN_FOLD FONTS
\usepackage[defaultfam,tabular,lining]{montserrat} %% Option 'defaultfam'
%% only if the base font of the document is to be sans serif
\usepackage[T1]{fontenc}
\renewcommand*\oldstylenums[1]{{\fontfamily{Montserrat-TOsF}\selectfont #1}}
% END_FOLD FONTS

% BEGIN_FOLD TITLE FORMATS
\titleformat{\chapter}[hang]{\Huge\bfseries\color{white}}{\thechapter \ \textasciitilde \  }{0pt}{\MakeUppercase{#1}}[\vspace{-1cm}]
\titleformat{\section}[hang]{\Large\bfseries\color{white}}{\thesection \ \textasciitilde \ }{0pt}{\MakeUppercase{#1}}
\titleformat{\subsection}[hang]{\large\bfseries\color{whitesmoke}}{\thesubsection \ \textasciitilde \ }{0pt}{\MakeUppercase{#1}}
% END_FOLD TITLE FORMATS

% BEGIN_FOLD PREFACE
\newcommand{\preface}{%
\newpage
\section*{PREFACE}
The purpose of this document is to act as a comprehensive note for my understanding on the subject matter. I may also use references aside from the lecture material to further organize my understanding, and these references will be listed under this portion.

In general this document follows the format of highlighting \keyword{keywords} in green. I can also introduce a {{\itshape\scshape\color{purple!55!red}{Definition}}} or a {{\itshape\scshape\color{red!55!purple}{Theorem}}}. There may also be various other things like code blocks which include \code{keywords} or \str{strings}. Remarks (similar to markdown style quotes). Or highlighted boxes. I might use these to organize things further if I deem necessary.

\section*{REFERENCES}
\begin{itemize}
	\item Provided Lecture Notes \& Info on Course Website
	\item Applied Digital Signal Processing: Theory and Practice - G. Manolakis, K. Ingle 
	\item Digital Signal Processing: Principles, Algorithms, and Applications 4e - G. Proakis, G. Manolakis
\end{itemize}
\newpage
}


% END_FOLD PREFACE

% BEGIN_FOLD COVERPAGE
\newcommand{\makecoverpage}[5]{
	
	%%%% LAYERS %%%%
	\thispagestyle{empty}
	\pgfdeclarelayer{bg}
	\pgfdeclarelayer{main}
	\pgfdeclarelayer{fg}
	\pgfsetlayers{bg, main, fg}
	
	%%% DRAWING %%%
	\begin{tikzpicture}[remember picture, overlay]
		
		\begin{pgfonlayer}{fg}
			\fill[greyblue] (current page.south west) rectangle (current page.north east);
		\end{pgfonlayer}
		
	\end{tikzpicture}
	
	\begin{tikzpicture}[remember picture, overlay]
		
		\begin{pgfonlayer}{fg}
			% title and subtitle
			\node[align=right, text=white, anchor=east] at ([xshift=10cm] current page.center)
			{\Huge\bfseries\fontsize{40}{40}\selectfont #5};
			\node[align=right, text=white, anchor=east] at ([xshift=10cm, yshift=2cm] current page.center)
			{\Huge\bfseries\fontsize{40}{40}\selectfont #1};
			\node[align=right, text=whitesmoke, anchor=east] at ([xshift=10cm,yshift=-1.5cm]current page.center) {\Large\item\fontsize{20}{20}\selectfont #2};
			
			% author and date 
			\node[align=right, anchor=south east] at ([xshift=-1cm, yshift=2cm]current page.south east) {\Large\color{whitesmoke}#3};
			\node[align=right,, anchor=south east] at ([xshift=-1cm, yshift=1cm]current page.south east) {\Large\color{whitesmoke}#4};
		\end{pgfonlayer}
		
	\end{tikzpicture}
	
	\begin{tikzpicture}[remember picture, overlay]
		\begin{pgfonlayer}{bg}
			% Define the number of sides and the radius
			\def\numSides{7}
			\def\radius{3cm}
			
			% Loop to draw concentric polygons
			\foreach \i in {1,...,15}{
				\node[draw, grey, dash pattern= on 1pt off 5+2*\i pt, line width = 1 pt, inner sep = 1cm, regular polygon, regular polygon sides=\numSides, minimum size=2*\i*\radius, rotate= 15+5*\i, opacity=50] at (-3,-6) {};
			}
		\end{pgfonlayer}
	\end{tikzpicture}
	\newpage
}
% END_FOLD COVERPAGE

% BEGIN_FOLD ENVIRONMENTS

%----------------------------------------------------------------------------------------
%   HIGHLIGHT ENVIRONMENT
%----------------------------------------------------------------------------------------

\newtcolorbox{highlight}{
	colback=greyblue,
	colframe=whitesmoke,
	coltext=whitesmoke,
	boxrule=0.75pt,
	boxsep=4pt,
	arc=0pt,
	outer arc=0pt,
	enlarge bottom by=0.25cm,
	enlarge top by=0.15cm
}

%----------------------------------------------------------------------------------------
%   REMARK ENVIRONMENT
%----------------------------------------------------------------------------------------

\newtcolorbox{remark}{
	colback=greyblue,
	colframe=whitesmoke,
	coltext=whitesmoke,
	boxrule=0pt,
	leftrule=0.75pt,
	boxsep=4pt,
	arc=0pt,
	outer arc=0pt,
	enlarge bottom by=0.5cm,
}

%----------------------------------------------------------------------------------------
%   OUTLINE ENVIRONMENT
%----------------------------------------------------------------------------------------

\newtcolorbox{outline}[1][4pt]{
	enhanced,
	colback=greyblue,
	colframe=greyblue, % To essentially hide the frame, but we will draw the corners manually
	coltext=whitesmoke,
	boxrule=1pt,
	boxsep=#1,
	arc=1pt,
	outer arc=1pt,
	enlarge bottom by=0.5cm,
	overlay={
		% Top left corner
		\draw[whitesmoke,line width=1pt] 
		(frame.north west) -- ++(0,-0.25*\tcbtextheight)
		(frame.north west) -- ++(0.25*\tcbtextwidth,0);
		% Bottom right corner
		\draw[whitesmoke,line width=1pt]
		(frame.south east) -- ++(0,0.25*\tcbtextheight)
		(frame.south east) -- ++(-0.25*\tcbtextwidth,0);
	}
}

%----------------------------------------------------------------------------------------
%   LSTLISTING ENVIRONMENT
%----------------------------------------------------------------------------------------
\lstset
{ %Formatting for code in appendix
	language=Verilog,
	frame=single,
	basicstyle=\footnotesize\texttt,
	numbers=left,
	stepnumber=1,
	showstringspaces=false,
	tabsize=4,
	breaklines=true,
	breakatwhitespace=false,
	aboveskip=2em,
	belowcaptionskip=0.75em
}
\usepackage[font={color=white, it}]{caption}
\renewcommand{\lstlistingname}{\color{white} Snippet}
\lstset{ %
	basicstyle=\footnotesize\ttfamily,        % size of fonts used for the code
	breaklines=true,                 % automatic line breaking only at whitespace
	captionpos=b,                    % sets the caption-position to bottom
	commentstyle=\color{grey},    % comment style
	escapeinside={\%*}{*)},          % if you want to add LaTeX within your code
	keywordstyle=\color{blue!55!green},       % keyword style
	stringstyle=\color{green!75!white},     % string literal style
}

%----------------------------------------------------------------------------------------
%   DEFINITION ENVIRONMENT
%----------------------------------------------------------------------------------------
\newenvironment{definition}{
	\begin{remark}\begin{itemize}[labelindent=5em,labelsep=0.25cm,leftmargin=*]
		\item[{{\itshape\scshape\color{purple!55!red}{Definition}}}]{}
	}
	{
	\end{itemize}\end{remark}
}

%----------------------------------------------------------------------------------------
%   THEORY ENVIRONMENT
%----------------------------------------------------------------------------------------
\newenvironment{theory}{
	\begin{itemize}[labelindent=4.2em,labelsep=0.5cm,leftmargin=*]
		\item[{{\itshape\scshape\color{red!55!purple}{Theorem}}}]
	}
	{
	\end{itemize}
}
%----------------------------------------------------------------------------------------
%   KEYWORD, CODE, EMPH & STROKE ENVIRONMENT
%----------------------------------------------------------------------------------------
\newcommand{\keyword}[1]{{{\color{green}{#1}}}}
\newcommand{\code}[1]{{\ttfamily\color{blue!55!green}\,{#1}\,}}
\newcommand{\str}[1]{{\ttfamily\color{green!55!white}\,{"#1"}\,}}
\renewcommand{\emph}[1]{} % renew emph style here as required
% END_FOLD ENVIRONMENTS

% BEGIN_FOLD PAGE NUMBS
\pagenumbering{gobble}
% END_FOLD PAGE NUMBS

% END_FOLD PREAMBLE
% BEGIN_FOLD DOCUMENT
\begin{document}	
	
\makecoverpage{Digital Systems}{ECE 3351}{Arnav Goyal}{Winter '23-'24}{\& Signal Processing}
\tableofcontents
\preface

\chapter{INTRODUCTION}

\section{SIGNALS}

\subsection*{SIGNALS \& SYSTEMS}
As we begin the study of \keyword{signal processing} we should start by defining what a \keyword{signal} actually is. 

\begin{definition}
	A \keyword{signal} is a measurable or detectable physical quantity (air-pressure, voltage, current, etc.) that carries some information.
\end{definition}

The study of signal processing involves passing a signal through a \keyword{system}. With the goal of extracting something (usually the information) from the signal
\begin{definition}
	A \keyword{system} is a physical device that performs an operation on a signal.
\end{definition}

\subsection*{INFORMATION \& NOISE}
Signals will always contain two things. \keyword{Information} and \keyword{Noise}. The information is the part we are interested in, and the Noise is considered everything else. For example when dealing with speech recognition, background music would be considered noise.

Information can be encoded within a signal through the some of the following methods:
\begin{itemize}
	\item Signal Amplitude
	\item Signal frequency or Spectral Content
	\item Signal Phase
\end{itemize}

Now we can see that signal processing deals with extracting information by essentially 'interpreting' its used encoding method.

\subsection*{ANALOG \& DIGITAL SIGNALS}

Signals can be found as \keyword{analog signals} or as \keyword{digital signals}.
\begin{definition}
	An \keyword{analog signal} is a signal that varies continuously with respect to its independent variables.
\end{definition}

Most real life signals are analog signals, and analog signals can either be processed directly or converted into digital signals for processing before being reconverted into analog signals.

\begin{definition}
	A \keyword{digital signal} is a signal that represents data as a sequence of discrete and quantized values. In other words, it is a discrete-valued AND discrete-time signal.
\end{definition}

Digital signals have some pretty big benefits:
\begin{itemize}
	\item Noise robustness - either a 0 or 1
	\item Software Implementation - can be processed through code instead of circuits
	\item Easy to store and transmit
	\item Fairly independent of external parameters like temperature
\end{itemize}

They have some drawbacks too:
\begin{itemize}
	\item Increased system complexity - Need A/D and D/A conversion and complex digital circuitry
	\item Limited Range of Frequencies - Sampling rate must be twice the value of the highest frequency present within the signal
	\item Power Consumption - requires active devices, rather than analog signal processors which can be passive.
\end{itemize}

Overall, the advantages matter more than the drawbacks which is why digital signals are so commonplace.

\section{CLASSIFICATION OF SIGNALS}

Signals can be further classified based on the information they provide. Namely to classify a signal we choose an option among the 5 characteristics below.
\begin{itemize}
	\item Continuous-Time OR Discrete-Time
	\item Continuous-Valued OR Discrete-Valued
	\item One-Channel OR Multi-Channel
	\item One-Dimensional OR Multi-Dimensional
	\item Deterministic OR Random
\end{itemize}

\subsection*{CONTINUOUS VS DISCRETE TIME SIGNALS}

\keyword{Continuous-time signals} are defined on some interval $(a,b)$, thus it can be represented as a function of a continuous variable $f\left( t \right) , t\in\left[ a,b\right]$

\keyword{Discrete-time signals} are defined on a set of integer numbers (indices) $\left\{ 0,1,2,\ldots \right\}$ and each index has a corresponding value $\left\{ f_0,f_1,f_2,\ldots \right\}$. This means that we can represent them as sequences: $f\left[ n\right] = \left\{ f_0,f_1,f_2,\ldots \right\}$

\subsection*{CONTINUOUS VS DISCRETE-VALUED SIGNALS}

\keyword{Continuous-valued signals} can take on all-possible values from its continuous range, while
\keyword{discrete-valued signals} take on values from a finite set. The process of converting a continuous-valued signal into a discrete-valued one is called \keyword{quantization}

\subsection*{SIGNAL CHANNELS}
\keyword{single channel signals} are usually generated by one source and can be represented by \emph{scalar} functions, \keyword{multi-channel signals} are usually generated by multiple sources/sensors and are represented by \emph{vector functions}

\subsection*{SIGNAL DIMENSIONS}

\keyword{One-dimensional signals} are a function of one independent variable (usually time). while \keyword{N-dimensional signals} are a function of N independent variables (time, x-pos, y-pos, etc.)

\subsection*{DETERMINISTIC VS RANDOM SIGNALS}

If we can exactly predict the future values of a signal by using its past values, we say that the signal is a \keyword{deterministic signal}. On the other hand, if we cannot predict it exactly we say that the signal is a \keyword{random signal}. There is no sharp distinction between the two due to the presence of noise, but these serve the purpose of broad categories.

\begin{highlight}
	Consider the signal fed into a colored TV. This signal would typically have 3-channels (R, G, B) and 3-dimensions - time, and coordinates to differentiate each pixel (t, x-pos, y-pos).
	
	\[ \mathbb{I}\left( x,y,t \right) = \begin{bmatrix}
		I_r\left(x,y,t\right) \\ I_g\left(x,y,t\right) \\ I_b\left(x,y,t\right)
	\end{bmatrix}	\]
	
\end{highlight}

\end{document}