% BEGIN_FOLD PREAMBLE
\documentclass[]{report}

%BEGIN_FOLD IMPORTS
\usepackage[]{tikz}
\usepackage{circuitikz}
\usetikzlibrary{shapes.geometric}
\usepackage[]{xcolor}
\usepackage[]{anyfontsize}
\usepackage[top=20mm, bottom=20mm, left=25mm, right=25mm, papersize={640 pt, 480 pt}]{geometry}
\usepackage[explicit]{titlesec}
\usepackage{fancyhdr}
\usepackage[most]{tcolorbox}
\usepackage{listings}
\usepackage[]{parskip}
\usepackage{caption}
\usepackage{enumitem}
\usepackage[hidelinks]{hyperref}
\hypersetup{linktoc=all}
\usepackage[outline]{contour}
\usepackage{soul}
\sethlcolor{greyblue!95!white}
% END_FOLD IMPORTS

\ctikzset{logic ports = ieee}
\tikzset{flipflop SR Latch/.style={flipflop,
		flipflop def={t1=$S$, t3=$R$, t5=$Q$}}
}

% BEGIN_FOLD COLORS
\definecolor{whitesmoke}{HTML}{CCCCCC}
\definecolor{grey}{HTML}{686E75}
\definecolor{red}{HTML}{D65B5C}
\definecolor{orange}{HTML}{EF8511}
\definecolor{yellow}{HTML}{D3B006}
\definecolor{green}{HTML}{5DB267}
\definecolor{blue}{HTML}{62ABF3}
\definecolor{purple}{HTML}{7F76DA}
\definecolor{pink}{HTML}{B76CC7}
\definecolor{greyblue}{HTML}{171B20} % default: 171B20

\pagecolor{black!95}
\AtBeginDocument{\color{white}}

% END_FOLD COLORS

% BEGIN_FOLD FONTS
\usepackage{tgpagella}
\usepackage[T1]{fontenc}
% END_FOLD FONTS

% BEGIN_FOLD TITLE FORMATS
\titleformat{\chapter}[hang]{\Huge}{}{0pt}{#1}[\vspace{-1cm}]
\titleformat{\section}[hang]{\Large\itshape}{}{0pt}{#1}
\titleformat{\subsection}[hang]{\large\bfseries}{\thesubsection \ \textasciitilde \ }{0pt}{#1}
% END_FOLD TITLE FORMATS

% BEGIN_FOLD PREFACE
\newcommand{\preface}{%
	\newpage
	\section*{\scshape PREFACE}
	The purpose of this document is to act as a comprehensive note for my understanding on the subject matter. I may also use references aside from the lecture material to further organize my understanding, and these references will be listed under this portion.
	
	In general this document follows the format of highlighting \keyword{keywords} in green. I can also introduce a {{\itshape\scshape\color{purple!55!red}{Definition}}} or a {{\itshape\scshape\color{red!55!purple}{Theorem}}}. There may also be various other things like code blocks which include \code{keywords} or \str{strings}, remarks (similar to markdown style quotes), or highlighted boxes. I might use these to organize things further if I deem necessary.
	
	\section*{\scshape REFERENCES}
	\begin{itemize}
		\item Provided Lecture Notes \& Info on Course Website
		\item Digital Design With An Introduction to the Verilog HDL, 5e - M. Mano, D. Ciletti
	\end{itemize}
	\newpage
}


% END_FOLD PREFACE

% BEGIN_FOLD COVERPAGE
\newcommand{\makecoverpage}[5]{
	
	%%%% LAYERS %%%%
	\thispagestyle{empty}
	\pgfdeclarelayer{bg}
	\pgfdeclarelayer{main}
	\pgfdeclarelayer{fg}
	\pgfsetlayers{bg, main, fg}
	
	%%% DRAWING %%%
	\begin{tikzpicture}[remember picture, overlay]
		
		\begin{pgfonlayer}{fg}
			\fill[greyblue] (current page.south west) rectangle (current page.north east);
		\end{pgfonlayer}
		
	\end{tikzpicture}
	
	\begin{tikzpicture}[remember picture, overlay]
		
		\begin{pgfonlayer}{fg}
			% title and subtitle
			\node[align=right, text=white, anchor=east] at ([xshift=10cm] current page.center)
			{\Huge\bfseries\fontsize{40}{40}\selectfont #5};
			\node[align=right, text=white, anchor=east] at ([xshift=10cm, yshift=2cm] current page.center)
			{\Huge\bfseries\fontsize{40}{40}\selectfont #1};
			\node[align=right, text=whitesmoke, anchor=east] at ([xshift=10cm,yshift=-1.5cm]current page.center) {\Large\item\fontsize{20}{20}\selectfont #2};
			
			% author and date 
			\node[align=right, anchor=south east] at ([xshift=-1cm, yshift=2cm]current page.south east) {\Large\color{whitesmoke}#3};
			\node[align=right,, anchor=south east] at ([xshift=-1cm, yshift=1cm]current page.south east) {\Large\color{whitesmoke}#4};
		\end{pgfonlayer}
		
	\end{tikzpicture}
	
	\begin{tikzpicture}[remember picture, overlay]
		\begin{pgfonlayer}{bg}
			% Define the number of sides and the radius
			\def\numSides{7}
			\def\radius{3cm}
			
			% Loop to draw concentric polygons
			\foreach \i in {1,...,15}{
				\node[draw, grey, dash pattern= on 1pt off 5+2*\i pt, line width = 1 pt, inner sep = 1cm, regular polygon, regular polygon sides=\numSides, minimum size=2*\i*\radius, rotate= 15+5*\i, opacity=50] at (-3,-6) {};
			}
		\end{pgfonlayer}
	\end{tikzpicture}
	\newpage
}
% END_FOLD COVERPAGE

% BEGIN_FOLD ENVIRONMENTS

%----------------------------------------------------------------------------------------
%   HIGHLIGHT ENVIRONMENT
%----------------------------------------------------------------------------------------

\newtcolorbox{highlight}{
	colback=greyblue,
	colframe=whitesmoke,
	coltext=whitesmoke,
	boxrule=0.75pt,
	boxsep=4pt,
	arc=0pt,
	outer arc=0pt,
	enlarge bottom by=0.25cm,
	enlarge top by=0.15cm
}

%----------------------------------------------------------------------------------------
%   REMARK ENVIRONMENT
%----------------------------------------------------------------------------------------

\newtcolorbox{remark}{
	colback=greyblue,
	colframe=whitesmoke,
	coltext=whitesmoke,
	boxrule=0pt,
	leftrule=0.75pt,
	boxsep=4pt,
	arc=0pt,
	outer arc=0pt,
	enlarge bottom by=0.5cm,
}

%----------------------------------------------------------------------------------------
%   OUTLINE ENVIRONMENT
%----------------------------------------------------------------------------------------

\newtcolorbox{outline}[1][4pt]{
	enhanced,
	colback=greyblue,
	colframe=greyblue, % To essentially hide the frame, but we will draw the corners manually
	coltext=whitesmoke,
	boxrule=1pt,
	boxsep=#1,
	arc=1pt,
	outer arc=1pt,
	enlarge bottom by=0.5cm,
	overlay={
		% Top left corner
		\draw[whitesmoke,line width=1pt] 
		(frame.north west) -- ++(0,-0.25*\tcbtextheight)
		(frame.north west) -- ++(0.25*\tcbtextwidth,0);
		% Bottom right corner
		\draw[whitesmoke,line width=1pt]
		(frame.south east) -- ++(0,0.25*\tcbtextheight)
		(frame.south east) -- ++(-0.25*\tcbtextwidth,0);
	}
}

%----------------------------------------------------------------------------------------
%   LSTLISTING ENVIRONMENT
%----------------------------------------------------------------------------------------
\lstset
{ %Formatting for code in appendix
	language=Verilog,
	frame=single,
	basicstyle=\footnotesize\texttt,
	numbers=left,
	stepnumber=1,
	showstringspaces=false,
	tabsize=4,
	breaklines=true,
	breakatwhitespace=false,
	aboveskip=2em,
	belowcaptionskip=0.75em
}
\usepackage[font={color=white, it}]{caption}
\renewcommand{\lstlistingname}{\color{white} Snippet}
\lstset{ %
	basicstyle=\footnotesize\ttfamily,        % size of fonts used for the code
	breaklines=true,                 % automatic line breaking only at whitespace
	captionpos=b,                    % sets the caption-position to bottom
	commentstyle=\color{grey},    % comment style
	escapeinside={\%*}{*)},          % if you want to add LaTeX within your code
	keywordstyle=\color{blue!55!green},       % keyword style
	stringstyle=\color{green!75!white},     % string literal style
}

%----------------------------------------------------------------------------------------
%   DEFINITION ENVIRONMENT
%----------------------------------------------------------------------------------------
\newenvironment{definition}{
	\begin{itemize}[labelindent=5em,labelsep=0.25cm,leftmargin=*]
		\item[{{\itshape\scshape\color{purple!55!red}{Definition}}}]{}
	}
	{
	\end{itemize}
}

%----------------------------------------------------------------------------------------
%   THEORY ENVIRONMENT
%----------------------------------------------------------------------------------------
\newenvironment{theory}{
	\begin{itemize}[labelindent=4.2em,labelsep=0.5cm,leftmargin=*]
		\item[{{\itshape\scshape\color{red!55!purple}{Theorem}}}]
	}
	{
	\end{itemize}
}
%----------------------------------------------------------------------------------------
%   KEYWORD, CODE, EMPH & STROKE ENVIRONMENT
%----------------------------------------------------------------------------------------
\newcommand{\keyword}[1]{{{\bfseries\color{purple}{#1}\,}}}
\newcommand{\code}[1]{{\ttfamily\color{blue!55!green}\,{#1}\,}}
\newcommand{\str}[1]{{\ttfamily\color{green!55!white}\,{"#1"}\,}}
\renewcommand{\emph}[1]{} % renew emph style here as required
% END_FOLD ENVIRONMENTS

% BEGIN_FOLD PAGE NUMBS
\pagenumbering{gobble}
% END_FOLD PAGE NUMBS

% END_FOLD PREAMBLE
% BEGIN_FOLD DOCUMENT
\begin{document}	
	
\tableofcontents
 
% \input{Review}

\chapter{Memory Mapping}

This chapter will discuss the useful concept of \keyword{memory mapping}, which is a technique used to assign the memory address space to different physical devices.

\section{Memory}

A \keyword{memory device} is a construct capable of storing a large quantity of binary information. Memory devices are made up of smaller cells.
\begin{itemize}
	\item There are two types of memory: \keyword{Random Access Memory (RAM)} \& \keyword{Read Only Memory (ROM)}
	\item  RAM supports the fundamental \keyword{read} and \keyword{write} operations; ROM just supports \keyword{read}.
\end{itemize}

\section{The Memory Cell}

\keyword{Memory cells} are the fundamental storage component present in a memory device. A 1-bit memory cell is called a \keyword{binary cell (BC)} and can be modeled by a D-latch and some additional gates. The binary cell has 4 ports: Input, Output, Read/Write, and Select.
\begin{itemize}
	\item The select signal acts as the \keyword{enable} signal of the cell
	\item The Read/Write signal meaning corresponds to \keyword{read = 1} and \keyword{write = 0}
	\item The input accepts data to be stored within the cell
	\item The output provides the data currently stored within the cell
\end{itemize}
The equivalent logic for a binary cell looks like the following: 
\\
\begin{center} \begin{circuitikz}
		\draw (2,0) node[flipflop SR Latch] (SR) {};
		\draw (SR.pin 1) node[and port, anchor=out, number inputs=3] (AND1) {};
		\draw (SR.pin 3) node[and port, anchor=out, number inputs=3] (AND2) {};
		\draw (AND1.in 3) to ++(-3,0) node[circ] (INPUT) {} to ++(-1,0) node[label=left:Input] {};
		\draw (AND2.in 3) to ++(-0.5,0) node[not port, anchor=out, scale=0.5] (NOT1) {};
		\draw (INPUT) |- (NOT1.in);
		\draw (AND1.in 2) to ++(-2.5,0) node[circ] (SELECT) {} to ++(-1.5,0) node[label=left:Select] {};
		\draw (AND2.in 2) to ++(-2.5,0) node[circ] to (SELECT);
		\draw (AND1.in 1) -| ++(-2,-3) node[not port, anchor=out, scale=0.5, rotate=90] (NOT2) {};
		\draw (NOT2.in) |- ++(0,-1) node[circ](READWRITE){} to ++(-2,-) node[label=left:Read/Write] {};
		\draw (SR.pin 5) to ++(1,0) node[and port, anchor = in 1, number inputs = 3] (AND3) {};
		\draw (AND2.in 1) to ++(-2,0) node[circ]{};
		\draw (AND2.in 2) ++(-2.5, 0) to ++(0,-2.5) -| ++(7,0) |- (AND3.in 2)
		\draw (READWRITE) to ++(7,0) |- (AND3.in 3)
		\draw (AND3.out) node[label=right:Output] () {};
\end{circuitikz} \end{center}
\vspace{1em}

This BC can now be used to create RAM by grouping many of these together.
\begin{itemize}
	\item BCs can be \keyword{grouped} by sharing select and read/write signals. This allows them to all be enabled at once, and read/write together.
	\item If we group $n$ BCs together, we can write $n$ bits of information and read $n$ bits of information by supplying an appropriate read/write and select signal.
	\item If we make multiple groups, and find a way to select only one group at a time (through a decoder) we have essentially created RAM.
\end{itemize} 
{\itshape Note:} This design is \keyword{not physically implementable!} When a BC is not selected it has an output of 0 (GND) onto the shared bus, we actually need \keyword{tri-state logic} to implement this with a shared bus. This example mainly serves the purpose of explaining the \keyword{equivalent logic} behind RAM, not actual implementation.

\section{Random Access Memory (RAM)}
Now that we have learned to create RAM, lets learn about it as a block. 

For starters, the time it takes to transfer information to and from any \keyword{random desired location} is always the same.

Consider a RAM with $k$-address lines, which correspond to a maximum of $2^k$ memory addresses/words (\# of groupings), and $n$-bits per word (\# of BCs per grouping). This RAM has 4 ports with the following specifications. We denote this as a $2^k \times n$ RAM.
\begin{itemize}
	\item $n$-bit input line
	\item $n$-bit output line
	\item $k$-bit address line
	\item $1$-bit read/write line
\end{itemize}
Because we have $n$ bits per group, and $2^k$ maximum groupings, our \keyword{memory capacity} is $2^k \times n$ bits. 

There are a couple \keyword{optimizations} we can do to make this RAM better, such as using 2 dimensional decoding instead of 1, and combining the input/output data lines into a shared bus using tri-state logic. However, this is not a digital design class, thus it is covered within the textbook.

All RAM is also not the same, there are two types: \keyword{Static RAM (SRAM}) and \keyword{Dynamic RAM (DRAM}).
\begin{itemize}
	\item SRAM consists of internal latches that store the information, information is retained as long as power is provided (\keyword{volatile memory}).
	\item DRAM stores the information as electric charge on capacitors within the chip through MOSFETS. This charge slowly leaves over time, and needs to be \keyword{periodically refreshed}.
\end{itemize}
We created SRAM earlier, and in general it is easier to use and has shorter read/write cycles. DRAM offers reduced power consumption and larger storage capacity within a single chip which is what makes it commonplace in industry.

\section{Read Only Memory (ROM)}

read-only-memory is memory device in which \keyword{permanent} binary information is stored.
\begin{itemize}
	\item Once stored, it stays within the unit even after the power is turned off. (\keyword{non-volatile memory})
	\item The $k$-inputs provide the address for the memory and the $n$-outputs provide the stored data-bits. Denoted as a $2^k \times n$ ROM.
	\item it is organized the same as the RAM we created, thus it has a maximum capacity of $2^k \times n$ bits.
\end{itemize}

Contrary to its naming convention, we actually can store data within the ROM. We do this by providing the \keyword{truth table} containing all possible memory addresses and stored data for each address, so that we can program it into the ROM. 







\end{document}
% END_FOLD DOCUMENT
