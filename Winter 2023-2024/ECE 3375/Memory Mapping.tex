\section{Memory Mapping}

This chapter is best learned through examples, and that is exactly what will be done here.

\subsection{Example 1}
Consider a Von Neumann memory mapped system with 4 KB of memory space (2K RAM + 2K ROM).

First we consider the bus width $w$ for the whole system by finding how many address lines we need to represent the entire memory space of 4 KB.
	\[ w = \frac{\ln{4000}}{\ln{2}} \approx 12	\]
This means that the system has a bus width of 12, and consequently occupies three hex digits from 0x000 to 0xFFF.

Lets find out the width required to specify a location onto the 2KB RAM and 2KB ROM. They have the same total space so they will have the same address width of 11.
	\[ r = \frac{\ln{2000}}{\ln{2}} \approx 11 \]
Thus we have one digit (The MSb) as a degree of freedom, which we will use to differentiate between selecting an address in the RAM or ROM. Lets consider the MSb $a_{11} = 0$ means selecting from RAM, and $a_{11}=1$ implies selecting from ROM.

Essentially the 12th bit is used to select from RAM or ROM, and the remaining 11 bits are used to select an address.



