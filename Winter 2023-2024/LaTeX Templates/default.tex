\documentclass[oneside, openany]{tufte-book}

\usepackage[]{lipsum}
\usepackage{circuitikz}
\usepackage{enumitem}
\usepackage[dvipsnames]{xcolor}
\usepackage{tcolorbox}
\usepackage{listings}
\setcounter{tocdepth}{1}
\usepackage{booktabs}
\usepackage{amsmath, xparse}
\usepackage{parskip}
\usepackage[font={color=black}]{caption}
\captionsetup{
	font={footnotesize}
}

\geometry{
	paperwidth=640pt,
	paperheight=480pt,
	right=7cm, left=25mm, bottom=25mm
}

\titlespacing*{name=\chapter,numberless}{0pt}{-35pt}{40pt}
\titlespacing*{name=\chapter}{0pt}{-22pt}{40pt}
\titleformat{\part}[block]{\normalfont\Huge\filcenter\color{darkgray}}{\partname\ \thepart}{2em}{} % Adjust the format as needed


\renewcommand{\maketitlepage}{%
{
		\begin{fullwidth}%
			\thispagestyle{empty}
			\fontsize{18}{20}\selectfont\par\noindent\textcolor{darkgray}{\itshape\thanklessauthor}%
			\vspace{11.5pc}%
			\fontsize{36}{40}\selectfont\par\noindent\textcolor{darkgray}{\thanklesstitle}%
			\vspace{2mm}
			\fontsize{14}{16}\selectfont\par\noindent\textcolor{darkgray}{\scshape\thanklesspublisher}%
		\end{fullwidth}%
}
}

\AtBeginDocument{\setlength{\parindent}{0cm}}

\newcommand{\preface}{
\newpage
\section*{Preface}
The purpose of this document is to act as a comprehensive note for my understanding on the subject matter. I may also use references aside from the lecture material to further develop my understanding, and these references will be listed here. 

This document should eventually serve as a standalone reference for learning or review of the subject matter. There is also a lot of organization within these documents, please refer to the table of contents within your PDF viewer for ease of navigation.

\section*{References}
\begin{itemize}
	\myreferences
\end{itemize}
}


%% NEEDS TO BE DEFINED IN MAIN DOC
% define title, author, publisher, and \def\myreferences for use in the preface

\let\cleardoublepage\clearpage

%%% -------------------
%%% |  CUSTOM COLORS  |
%%% -------------------

\definecolor{whitesmoke}{HTML}{CCCCCC}
\definecolor{grey}{HTML}{686E75}
\definecolor{red}{HTML}{D65B5C}
\definecolor{orange}{HTML}{EF8511}
\definecolor{yellow}{HTML}{D3B006}
\definecolor{green}{HTML}{5DB267}
\definecolor{blue}{HTML}{62ABF3}
\definecolor{purple}{HTML}{7F76DA}
\definecolor{pink}{HTML}{B76CC7}
\definecolor{greyblue}{HTML}{171B20} % default: 171B20

%%% -------------------
%%% | CUSTOM COMMANDS |
%%% -------------------


%----------------------------------------------------------------------------------------
%   HIGHLIGHT ENVIRONMENT
%----------------------------------------------------------------------------------------

\newtcolorbox{highlight}{
	colback=greyblue,
	colframe=whitesmoke,
	coltext=whitesmoke,
	boxrule=0.75pt,
	boxsep=4pt,
	arc=0pt,
	outer arc=0pt,
	enlarge bottom by=0.25cm,
	enlarge top by=0.15cm
}

%----------------------------------------------------------------------------------------
%   REMARK ENVIRONMENT
%----------------------------------------------------------------------------------------

\newtcolorbox{remark}{
	colback=greyblue,
	colframe=whitesmoke,
	coltext=whitesmoke,
	boxrule=0pt,
	leftrule=0.75pt,
	boxsep=4pt,
	arc=0pt,
	outer arc=0pt,
	enlarge bottom by=0.5cm,
}

%----------------------------------------------------------------------------------------
%   OUTLINE ENVIRONMENT
%----------------------------------------------------------------------------------------

\newtcolorbox{outline}[1][4pt]{
	enhanced,
	colback=greyblue,
	colframe=greyblue, % To essentially hide the frame, but we will draw the corners manually
	coltext=whitesmoke,
	boxrule=1pt,
	boxsep=#1,
	arc=1pt,
	outer arc=1pt,
	enlarge bottom by=0.5cm,
	overlay={
		% Top left corner
		\draw[whitesmoke,line width=1pt] 
		(frame.north west) -- ++(0,-0.25*\tcbtextheight)
		(frame.north west) -- ++(0.25*\tcbtextwidth,0);
		% Bottom right corner
		\draw[whitesmoke,line width=1pt]
		(frame.south east) -- ++(0,0.25*\tcbtextheight)
		(frame.south east) -- ++(-0.25*\tcbtextwidth,0);
	}
}

%----------------------------------------------------------------------------------------
%   LSTLISTING ENVIRONMENT
%----------------------------------------------------------------------------------------
\lstset
{ %Formatting for code in appendix
	language=Verilog,
	frame=single,
	basicstyle=\footnotesize\texttt,
	numbers=left,
	stepnumber=1,
	showstringspaces=false,
	tabsize=4,
	breaklines=true,
	breakatwhitespace=false,
	aboveskip=2em,
	belowcaptionskip=0.75em
}
\renewcommand{\lstlistingname}{\color{black} Code Snippet}
\lstset{ %
	basicstyle=\footnotesize\ttfamily,        % size of fonts used for the code
	breaklines=true,                 % automatic line breaking only at whitespace
	captionpos=b,                    % sets the caption-position to bottom
	commentstyle=\color{grey},    % comment style
	escapeinside={\%*}{*)},          % if you want to add LaTeX within your code
	keywordstyle=\color{blue!55!green},       % keyword style
	stringstyle=\color{green!75!white},     % string literal style
}

%----------------------------------------------------------------------------------------
%   DEFINITION + THEORY ENVIRONMENT
%----------------------------------------------------------------------------------------
\newenvironment{definition}{
	\begin{itemize}[labelindent=5em,labelsep=0.25cm,leftmargin=*]
		\item[{{\itshape\scshape\color{purple!55!red}{Definition}}}]{}
	}
	{
	\end{itemize}
}


\newenvironment{theory}{
	\begin{itemize}[labelindent=4.2em,labelsep=0.5cm,leftmargin=*]
		\item[{{\itshape\scshape\color{red!55!purple}{Theorem}}}]
	}
	{
	\end{itemize}
}

\newenvironment{bullets}{
	\begin{itemize}[leftmargin=2em]
	}
	{
	\end{itemize}
}
%----------------------------------------------------------------------------------------
%   KEYWORD, CODE, EMPH & STROKE ENVIRONMENT
%----------------------------------------------------------------------------------------
\newcommand{\keyword}[1]{{{\bfseries\color{purple}{#1}\,}}}
\newcommand{\code}[1]{{\ttfamily\color{blue!55!green}\,{#1}\,}}
\newcommand{\str}[1]{{\ttfamily\color{green!55!white}\,{"#1"}\,}}
\renewcommand{\emph}[1]{{\itshape#1}} % renew emph style here as required
% END_FOLD ENVIRONMENTS
