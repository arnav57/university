\chapter{Social Inequality \& Social Class}

This chapter holds the content from week 3. The assigned reading is$\ldots$
\begin{bullets}
	\item Imagining Sociology - Chapter 4
\end{bullets}

\section{Capitalism}

Karl Marx was an academic in the social science world who thought that capitalism would eventually lead to socialism, and that socialism would eventually lead to communism.

In the eyes of Marx, \keyword{capitalism} is made up of two main \keyword{classes} of people:
\begin{bullets}
	\item The \keyword{bourgeoisie} (capitalists) are the people who own the means of production and property. They essentially own the means of producing more wealth.
	\item The \keyword{proletariat} (workers) do not own the means of production, they only own their capacity to labour\sidenote{mental or physical} which they must sell to the capitalist.
\end{bullets}

There also exist two other classes called the \keyword{petite-bourgeoisie}\sidenote{a small-scale business owner (coffee shop, corner store). They work alongside the proletariat and are more receptive to their needs} and the \keyword{lumpenproletariat}\sidenote{slum worker, the lowest layer of the working class, includes petty criminals and the jobless}.
\begin{bullets}
	\item Marx argued that eventually the petite-bourgeoisie would all become proletariat.
\end{bullets}


Marx also argued that capitalism was a \textit{"dictatorship of the bourgeoisie"}. He also denoted that they must coexist, and that either cannot survive without the other.

Marx argued that the core struggle in all societies is \keyword{class struggle}. Despite the co-dependent nature of both classes in capitalist society, the relationship between both is unequal, which can lead to class struggles.
\begin{bullets}
	\item The bourgeoisie wants to maximize \keyword{surplus value}\sidenote{profit}, this is most easily done by paying workers low wages, expecting fast work, and setting long hours.
	\item The proletariat want good wages, safe work, and a reasonable number of hours.
\end{bullets}
These interests are inherently conflicting.

Marx is clearly against capitalism as a whole, but he often wondered why the multitudes of proletariat dont band together to rise up against the few bourgeoisie. He argued that this didnt happen due to the role of \keyword{ideology}.
\begin{bullets}
	\item Ideologies are sets of conscious (or unconscious) ideas/beliefs that govern and guide people's lives.
	\item Marx noted that the dominant ideologies of any epoch\sidenote{time-period} are those of the dominant class in that period.
\end{bullets}

Capitalist society has a few ideologies: \keyword{meritocracy}\sidenote{Meritocracy is the idea that people will achieve based on their own merit. The idea that rich people MUST have worked harder than poor people to acquire their wealth.}, individualism, progress, expansion, etc.
\begin{bullets}
	\item We can see (from a Marxist lens) that the idea of meritocracy benefits the bourgeoisie the most - it legitimizes the fact that they have money, and highlights the hard-work it (might not have) took to get there.
	\item It also makes it seem that anyone can achieve their status. It encourages the belief that we can become like that if we work hard enough. But, it discounts the connections, social position, education, and other opportunities the rich have that could have gotten them to that status.
\end{bullets}

Marx believed that workers in capitalist society develop a \keyword{false consciousness} - i.e they support the ideologies that support the ruling-class (bourgeoisie), but hurt their own class. Ideas like meritocracy and individualism are all taught in social institutions (schools, mass media, family) to perpetuate them.

\section{Class Consciousness}
\keyword{Class consciousness} is a Marxist term used to refer to people's beliefs regarding their social class and class interests. It is an awareness of what is in the best interest of one's class and is an important precondition for organizing into a \textit{"class for itself"}\sidenote{a class organized in pursuit of its own interests} to advocate for class interests.
\begin{bullets}
	\item Marx wanted the working class to develop class consciousness
	\item \keyword{Unions} are organizations of employees who work together to negotiate better pay, benefits, hiring/firing practices etc. Unions are an example of a \textit{"class for itself"}
	\item \keyword{Trade Union Density} is the percentage of wage earners in a population that are part of a union. In 2018 Canada had a 26\%, this was down 3\% since 1998. In general this trend is on the decline and Canada had a very slight decline.
\end{bullets}

\section{The Three Bases of Power}
 
Max Weber took a slightly different approach to conceptualizing inequality. He began his study by defining power
\begin{bullets}
	\item \keyword{Power} is the chance of a man (or number of men) to realize their own will in a communal action, even against the resistance of others who are participating in the same action.
\end{bullets}

He then argued that there are three primary bases of power in society: \keyword{economic class}, \keyword{social status}, and \keyword{party}.
\begin{bullets}
	\item  Marx and Weber both defined the \textit{class} of an individual based on their relationship to the economy. Weber's definition of a class is a group of people sharing a common situation in the market, and having common interests.
	\item While Marx identified two main groups - Weber identified 4 main groups\sidenote{large capitalists, small capitalists, specialists, and the working class}
\end{bullets}

Weber's first base of power - economic class includes distinctions, the four main groups\sidenote[12]{}
\begin{bullets}
	\item Large capitalists - owners of large corporations; \textit{Marx's bourgeoisie}
	\item Small capitalists - owners of small businesses; \textit{Marx's petite-bourgeoisie}
	\item specialists - doctors, lawyers, professors have marketable skills to be sold
	\item working class - manual laborers; \textit{Marx's proletariat}
\end{bullets}

Weber's second base of power is status. A \keyword{status group} is one with a \textit{"style of life"} based on social honors and prestige that is expressed in their interactions with one another.
\begin{bullets}
	\item This can be formal - such as when we refer to someone as "doctor" or "lawyer" - it denotes a particular type of education and this title cannot simply be given to anyone
	\item This can be informal - such as respecting people who are older than you.
\end{bullets}

In general, people of high class tend to have high status. For example a CEO. Those with low social class tend to have low status. It is important to note that this is a correlation \textit{not a causation}. A priest is an example that breaks this rule. They have immense status, but not a very high class. On the other end, tradesman are probably of higher class, but do not have much status (societal respect).

Weber's third base of power is \keyword{party}. They are organizations that attempt to influence social action and focus on achieving some goal in the sphere of power. These need not be political parties, they can also be a neighborhood watch or parent-teacher group.

\section{Income Inequality in Canada}

Sociologists examine income inequality in Canada through a concept called \keyword{socio-economic status (SES)}. SES is a measure of an individual's of family's social and economic position relative to others. It is a composite scale that includes: income, education, and occupational prestige.
\begin{bullets}
	\item SES incorporates both Marx's and Weber's ideas into one concept.
\end{bullets}
This can deal with anomalies such as the priest or plumber - with only one high trait.

\keyword{Social mobility} is the movement on a stratification system (such as the class system). In general if there is a lot of it, Income inequality is less of a concern. 
\begin{bullets}
	\item Suppose the rich are getting richer by working harder, and suppose people can move from one class to another through means of hard work. This idea is an example of an \keyword{achievement-based stratification system}.
	\item Conversely, an \keyword{ascription-based stratification system} determines an individuals rank by his attained characteristics (sex, race, skin color, height, etc.)
\end{bullets}

Overall we are all born with a SES from our parents, but the extent to which a society has achievement based stratification, or ascription based stratification all depends on its level of social mobility.

Many studies compare social mobility across contries. Some measure \keyword{intergenerational income elasticity} - the statistical relationship between a parent's and child's economic standings. The higher the number, the less social mobility is present in a society.

\section{Poverty}

\keyword{Poverty} is a state in which resources are lacking. We can think about this issue in terms of \keyword{relative poverty} or \keyword{absolute poverty}
\begin{bullets}
	\item relative poverty is the deprivation of one person in comparison to another.
	\item absolute poverty is the life-threatening deprivation of an individual.
	\item Approx one billion people a year live in absolute poverty
\end{bullets}

To combat poverty, the government of Canada issued the \keyword{Canada Pension Plan (CPP)}. Essentially the CPP provides you with a pension (money after avg retirement at 65). This drastically reduced poverty after its launch in 1976.

The \keyword{cycle of poverty} refers to how poverty tends to self-perpetuate.
\begin{bullets}
	\item If you are poor and looking for a job, it is hard to print a resume, attach an address onto a resume, or have clean clothes to wear during the interview.
	\item Living in poverty is surprisingly MORE expensive than living normally - you might have to pay daily rates at a hotel while homeless, or eat take out every day because you have no kitchen to cook in
\end{bullets}

