\chapter{The Sociological Imagination}

This chapter holds weeks 1-2 worth of course content, and assigned readings listed below.
\begin{bullets}
	\item Imagining Sociology - Preface
	\item Imagining Sociology - Chapter 1
\end{bullets}

\section{Sociology}

The term \keyword{sociology} was coined by Auguste Comte
\begin{bullets}
	\item Comte sought to understand how society worked, and the effect of the larger processes on society and the people living in it.
	\item \keyword{Society} is the largest scale human group that shares common land and common institutions.\sidenote{Canada is home to two societies - Quebec, and everything else}
\end{bullets}

Society is based on social interactions among its members, this act is called \keyword{socializing}.
\begin{bullets}
	\item Through socialization we learn the written and unwritten rules of society
	\item The fact that most interactions in society are predictable\sidenote{Think of small talk with a cashier, you wouldn't truthfully answer the question "how are you?"} establishes a common set of understandings of how society works, and how we behave in it
	\item These rules can change over time, consider how different the a cashier interaction was before and during covid-19
\end{bullets}

Interactions in society are shaped by \keyword{culture}, a system of behavior, beliefs, knowledge, practices, values and materials.
\begin{bullets}
	\item The \keyword{dominant culture} is able to impose its traits onto a society.\sidenote{In america this tends to be: look good, be rich, and own a big house}
	\item A \keyword{counterculture} is a group that rejects certain elements of the dominant culture ex. anti-consumerist groups
	\item \keyword{Subcultures} also differ (but not needingly oppose) the dominant culture
\end{bullets}

Culture as a whole, is often divided into \keyword{high culture} and \keyword{low culture}.
\begin{bullets}
	\item High culture is the culture of societies elite (me), it might be difficult to appreciate unless one has been taught to enjoy and understand it (like i have)
	\item Low culture is the culture of the majority (you).
	\item Classical Music would be an example of high culture, while pop/rap would be low culture.
\end{bullets}

\section{The Sociological Imagination}

The \keyword{sociological imagination} is a book written by C. Wright Mills. It essentially highlights the inability to understand our own lives or understand the larger society independently from one another. He argued that in order to learn about one, we must really learn about both\sidenote{In the textbook this is referred to as connecting biography with society}.
\begin{bullets}
	\item Essentially we should try connecting \keyword{personal troubles} and \keyword{public issues}, in order to more deeply understand both of them.
\end{bullets}

There are three main areas (foci) of study within sociology include \keyword{social inequality}, \keyword{social institutions}, and \keyword{social change}
\begin{bullets}
	\item Social Inequality focuses on the gap between advantaged and disadvantaged people in society, based on the consequential differences between people, on the lives they lead.
	\item Social institutions are the norms, values, and rules of conduct that structure human interactions. They need not be buildings. In Canada there exist: The family, education, religion, the economy, and the government.
\end{bullets}