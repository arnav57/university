\chapter{Sociological Analysis}

\section{Fair?}
This course is titled "Life is not always \emph{fair}", However, what is actually considered fair? Here are some common things people consider to be fair:
\begin{bullets}
	\item \keyword{Equality}\sidenote{Maybe treating everyone as equals is considered fair, but what about addressing peoples individual circumstances?}
	\item \keyword{Equity}\sidenote{Maybe we should add some exceptions, and treat people based on circumstance, is this more fair?}
\end{bullets}

Regardless, here are working definitions that helps us wrap our heads around equality and equity

\begin{definition}
	\keyword{Gender equality} means that people of all genders enjoy the same status and have \emph{equal opportunity} to realize their full human rights, to contribute to national, economic, social, and cultural developments; and to also benefit from the results of those developments.
\end{definition}

\begin{definition}
	\keyword{Gender equity} is about being fair to people of all gender identities. It is the process of remedying historical and social oppression that would otherwise prevent people from fully contributing to political, cultural, and social life - and enjoying the benefits this contribution brings
\end{definition}

Essentially, equality involves the concept of equal opportunity, and equity is about remedying the underlying factors that contribute to the inequality.

\section{Not Always?}

The next part we will analyze is the concept of life \emph{not always} being fair. Well the term "always" refers to the frequency of this happening.
\begin{bullets}
	\item In social science we rarely speak in absolutes
\end{bullets}
The big question we are trying to ask is $\ldots$

If life is not always fair, Is it \emph{ever} fair? What would it take to make things more fair? (In whatever way that \emph{fair} means.)

\section{Life?}
By the term \emph{life} sociologists mean \keyword{social life}
\begin{bullets}
	\item The natural world is not inherently fair nor unfair, it just is - we all die eventually.
	\item fairness is a \emph{social concept} - subject to collective action
\end{bullets}
The big question here is, "Under what conditions shall we live and die with one another?"

\section{Life is Not Always Fair}
Now lets talk about the bigger picture. What is the actual context of a statement like, "Life is not always fair".
\begin{bullets}
	\item Capitulation?\sidenote{"Suck it up, it is what it is" mindset}
	\item Cynicism?\sidenote{Play dirty since everyone else is}
	\item Critique?\sidenote{Empirical challenge to the myth of "meritocracy" - the idea that people will achieve based on their own merit}
\end{bullets}

Meritocracy Implies the following
\begin{bullets}
	\item Failing to achieve is caused by a lack of positive behaviors.
	\item Individual choice determines social outcomes
\end{bullets}
Sociology invites us to consider factors beyond individual choices.

\section{Challenging the Idea of Meritocracy}
Lets try this by posing a question, "Is hockey success based on meritocracy?" In theory... Yes.
\begin{bullets}
	\item You need to be a good player
	\item Success requires: Talent + Hard Work
	\item You cant "buy" hockey achievement
\end{bullets}
Someone (with too much time)\sidenote{His name is Malcolm Gladwell, and yes, I did call him jobless} did a study on the birth months of hockey players, and there is a trend that suggests that the most successful players are born in the months of January to April, this suggests \emph{against} meritocracy

A possible societal explanation for this trend is as follows:
\begin{bullets}
	\item Eligibility cut-off for age-class hockey is Jan 1
	\item Those born earlier on have had more time to grow bigger and stronger than those born late that same year.
\end{bullets}

Its interesting that this trend carries all the way onto the later stages of life, surely the people born later can catch up with hard work? There must be other social factors then...
\begin{bullets}
	\item Advantages lead to even more advantages
	\item There is a social process that operates independent of anyone's individual choices. Selection\sidenote{Being older lets u be selected earlier for advanced teams} , Streaming\sidenote{You are streamed into better teams, with better teammates, better coaches, better opponents}, Differentiated Experiences\sidenote{ All of these things enable people to improve as a player, these constant achievements enables them to identify themselves as a "hockey player"}.
\end{bullets}
So will being born within this month range, make you Sidney Crosby? Obviously not, The key idea here is that $\text{Correlation}\neq\text{Causation}$\sidenote{Just because hockey success and birth months are correlated, it does not mean that birth month is the cause for hockey success}
