\chapter{Globalization \& Global Inequality}

This chapter contains week 4 of course content. The assigned readings are:
\begin{bullets}
	\item Imagining Sociology - Chapter 5
\end{bullets}

\section{Globalization}

\keyword{Globalization} is the process of increasing the interconnectedness of people, products, ideas and places. Globalization increased interconnectedness in three main ways:
\begin{bullets}
	\item Material and Physical Connections increase
	\item Making the world feel smaller, More like a \keyword{global village}\sidenote{We can easily interact with and learn about people from far away places through the internet}
	\item Dissemination of ideas/culture throughout the world
\end{bullets}

The world is generally becoming more interconnected, but some countries still prefer to be isolated. Why is this? For some countries globalization can lead to:
\begin{bullets}
	\item A waning of culture
	\item Increased unemployment
	\item A loss of autonomy over services and resources
\end{bullets}
In general globalization is not a linear process, it involves advances and regressions and can take a while.

\section{Modernization Theory}

\keyword{Modernization Theory} attempts to isolate the features that predict which societies will progress and develop. It argues that a societies economic, social, and cultural systems can either help of hinder development.
\begin{bullets}
	\item It claims that countries are poor because they cling to traditional and inefficient attitudes, tech, and institutions.
	\item The belief that, with enough time and help of \textit{correct} behaviors, all societies can become modernized like the western ones.
\end{bullets}

This modernization process requires that societies go through a set of established stages\sidenote{This is according to \textit{Rostow's Modernization Theory} or\textit{ Rostow's stages of growth}}.
\begin{enumerate}
	\item Traditional Society
	\item Preconditions for Takeoff
	\item Takeoff
	\item Drive to Maturity
	\item Age of Mass Consumption
\end{enumerate}

We start with the \keyword{traditional society} that prioritizes stability and sustaining themselves (mainly through agriculture). Eventually though, the demand for raw materials increases and these societies cannot keep up, we enter the second period
\begin{bullets}
	\item Need to innovate commercial agriculture; start selling \keyword{cash crops}
	\item Tech advances help this - irrigation systems, transport, etc.
	\item This leads to increased productivity and higher social mobility
\end{bullets}
This second period is called the \keyword{preconditions for takeoff}

The third period is (economic) \keyword{takeoff}:
\begin{bullets}
	\item Manufacturing becomes more efficient, increasing in size and scale, these goods can now be exported.
	\item Markets emerge as people produce goods to trade for profit.
	\item There is a rise of individualism, this can undermine family ties and timely norms and values.
\end{bullets}

Next, societies move into the \keyword{maturity} stage:
\begin{bullets}
	\item All sectors of society are involved in market production; International trade rises
	\item Economies diversify, Cities grow, Social movements start
	\item All this also results in a decrease in absolute poverty
\end{bullets}

Finally, we enter the \keyword{period of mass consumption}. Canada is currently in this stage.
\begin{bullets}
	\item The mass production from the last stage facilitates this stage
	\item People feel they need that new diversity of available goods, and consume them accordingly
	\item Consumers can do this due to their higher (and more disposable) incomes
\end{bullets}

An example of how agriculture has changed would be the practice of \keyword{monocropping}\sidenote{Which is an economically efficient and profitable method of repeatedly growing one high-yield crop}.

Modernization Theory is criticized for being \keyword{ethnocentric}, essentially judging other cultures by the standards of your own culture. Is it fair to measure other countries by western standards?