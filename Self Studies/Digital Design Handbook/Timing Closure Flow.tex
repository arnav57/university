\chapter{Synthesis \& Timing Closure Flow}

\section{Synthesis}

\keyword{Synthesis} is the process of converting Verilog or any code written in an HDL to equivalent logic, described by a \keyword{netlist}. An EDA tool\sidenote{such as ModelSim} is responsible for doing this and mapping it onto an FPGA device. 
\begin{bullets}
	\item We can also provide constraints to guide this process: setup/hold time constraints, power or area constraints, etc.
\end{bullets}

\section{Timing Closure Flow}

\keyword{Timing closure flow} is the process of ensuring that a design meets timing requirements. Digital circuits must meet their timing requirements in order to function.
\begin{bullets}
	\item If the propagation delay of some logic in between two flip-flops is longer than the clock period, the data will not arrive in time to be sampled at the clock pulse, rendering the circuit useless.
\end{bullets}

In order to avoid this, we employ the timing closure flow.
\begin{bullets}
	\item We start by specifying timing constraints, such as the desired clock frequency and setup/hold times.
	\item Next we test our design and see if it violates any timing constraints.
	\item If it does we examine the \keyword{critical paths}\sidenote{The paths that violate the timing constraints} and try to optimize them by tweaking logic or some other modifications.
	\item We then repeat those two steps until we meet our timing constraints while ensuring that any tweaks we made still retain functional integrity of the design.
\end{bullets}