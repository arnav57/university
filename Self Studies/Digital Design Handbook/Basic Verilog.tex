\chapter{The Basics}

\section{Hardware Description Languages}

A \keyword{hardware description language (HDL)} is a specialized computer language used to describe the nature of digital electronic circuits.
\begin{itemize}
	\item They include the notion of \keyword{time}\sidenote{They include a notion of time through gate delays, and how long it takes for the signal to propagate through them} and \keyword{concurrency}\sidenote{Concurrency allows multiple things to happen at the same time, this is different to a regular programming language like python which is inherently not concurrent and is sequential in nature}
\end{itemize}
In this handbook we will be concerning ourselves with the \keyword{Verilog} and \keyword{SystemVerilog} HDL. Both of which are used ubiquitously within the silicon industry for logic design and verification.
\begin{itemize}
	\item SystemVerilog is a superset of Verilog\sidenote{This is similar to the relationship between C and C++, we will also start by learning Verilog, and then diving into the features that SystemVerilog offers}
\end{itemize}

\section{Levels of Abstraction}
Within this field of study we will often deal with the concept of \keyword{abstraction}. Verilog offers the description of things at three levels of abstraction.
\begin{itemize}
	\item Gate-Level-Modeling\sidenote{This is the lowest-level of abstraction and allows us to manually code each gate, and wire. Verilog already has the logic gates ready for use through basic syntax}
	\item Dataflow Modeling\sidenote{Also called Register Transfer Level Modeling, here we can talk about how data flows through our circuit, and it deals with continuous assignment}
	\item Behavioural Modeling\sidenote{This allows us to describe the operation of our circuits in english-like words, this is the highest level of abstraction}
\end{itemize}
Here are some basic code examples of the three modeling styles $\ldots$

\begin{lstlisting}[caption={Gate-Level Modeling}]
and G1(out, A, B);
or G2(out, A, B);
nand G3(out, A, B);
\end{lstlisting}

\begin{lstlisting}[caption={RTL Modeling}]
assign out1 = x & y;
assign out2 = x | y;
assign out3 = ~y;
\end{lstlisting}

\begin{lstlisting}[caption={Behavioral Modeling}, label={mux2to1}]
always @(sel, I0, I1):
begin
	if (sel)
		out = I1;
	else
		out = I0;
end
\end{lstlisting}

\section{Modules \& Entities}

A \keyword{module} is the basic building block of Verilog, Modules are abstracted and interact with the external environment through their \keyword{ports}.
\begin{itemize}
	\item Modules can be an element, or a collection of other (lower-level) modules
	\item Modules can be instantiated (but cannot be defined!) within other modules.
	\item A module that is not instantiated within any other module, is referred to as the \keyword{top-level module}
\end{itemize}

\begin{marginfigure}
\begin{lstlisting}[caption={Module Declaration Syntax}, label={defmodule}]
module <moduleName> (
[port-list]
);
// Module specification goes here!
endmodule
\end{lstlisting}
\end{marginfigure}

It is quite easy to declare a module as shown in Code Snippet \ref{defmodule}, lets also learn by example though. Consider creating a 4$\times$2 MUX from two 2$\times$1 MUX blocks, which we have defined in Code Snippet \ref{mux2to1}. Lets assume that we encased that code in a module named \code{mux\_2x1}.

\begin{minipage}{\linewidth}
\begin{lstlisting}[caption={Creating a 4x2 MUX from two 2x1 MUX modules}, label={mux4to2}]
module mux_4x1(
	input i0, i1, i2, i3, sel0, sel1,
	output out
);

// wires to route internal connections
wire outA, outB;

// connection logic
2x1mux A (i0,i1,sel0, outA);
2x1mux B (i2,i3,sel0, outB);
2x1mux C (outA, outB, sel1, out);

endmodule
\end{lstlisting}
\end{minipage}