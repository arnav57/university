% BEGIN_FOLD PREAMBLE
\documentclass[]{report}

%BEGIN_FOLD IMPORTS
\usepackage[]{tikz}
\usepackage{circuitikz}
\ctikzset{logic ports = ieee}
\usetikzlibrary{shapes.geometric, decorations.markings}
\usepackage[]{xcolor}
\usepackage[]{anyfontsize}
\usepackage[top=20mm, bottom=20mm, left=25mm, right=25mm]{geometry}
\usepackage[explicit]{titlesec}
\usepackage{fancyhdr}
\usepackage[most]{tcolorbox}
\usepackage{listings}
\usepackage[]{parskip}
\usepackage{caption}
\usepackage{enumitem}
\usepackage[hidelinks]{hyperref}
\hypersetup{linktoc=all}
\usepackage[outline]{contour}
\usepackage{soul}
\sethlcolor{greyblue!95!white}
% END_FOLD IMPORTS

% BEGIN_FOLD COLORS
\definecolor{whitesmoke}{HTML}{CCCCCC}
\definecolor{grey}{HTML}{686E75}
\definecolor{red}{HTML}{D65B5C}
\definecolor{orange}{HTML}{EF8511}
\definecolor{yellow}{HTML}{D3B006}
\definecolor{green}{HTML}{5DB267}
\definecolor{blue}{HTML}{62ABF3}
\definecolor{purple}{HTML}{7F76DA}
\definecolor{pink}{HTML}{B76CC7}
\definecolor{greyblue}{HTML}{171B20} % default: 171B20

\pagecolor{greyblue}
\AtBeginDocument{\color{whitesmoke}}


% END_FOLD COLORS

% BEGIN_FOLD FONTS
\usepackage[defaultfam,tabular,lining]{montserrat} %% Option 'defaultfam'
%% only if the base font of the document is to be sans serif
\usepackage[T1]{fontenc}
\renewcommand*\oldstylenums[1]{{\fontfamily{Montserrat-TOsF}\selectfont #1}}
% END_FOLD FONTS

% BEGIN_FOLD TITLE FORMATS
\titleformat{\chapter}[hang]{\Huge\bfseries\color{white}}{\thechapter \ \textasciitilde \  }{0pt}{\MakeUppercase{#1}}[\vspace{-1cm}]
\titleformat{\section}[hang]{\Large\bfseries\color{white}}{\thesection \ \textasciitilde \ }{0pt}{\MakeUppercase{#1}}
\titleformat{\subsection}[hang]{\large\bfseries\color{whitesmoke}}{\thesubsection \ \textasciitilde \ }{0pt}{\MakeUppercase{#1}}
% END_FOLD TITLE FORMATS

% BEGIN_FOLD PREFACE
\newcommand{\preface}{%
	\newpage
	\section*{PREFACE}
	The purpose of this document is to act as a comprehensive note for my understanding on the subject matter. I may also use references aside from the lecture material to further organize my understanding, and these references will be listed under this portion.
	
	In general this document follows the format of highlighting \keyword{keywords} in green. I can also introduce a {{\itshape\scshape\color{purple!55!red}{Definition}}} or a {{\itshape\scshape\color{red!55!purple}{Theorem}}}. There may also be various other things like code blocks which include \code{keywords} or \str{strings}. Remarks (similar to markdown style quotes). Or highlighted boxes. I might use these to organize things further if I deem necessary.
	
	\section*{REFERENCES}
	\begin{itemize}
		\item Provided Lecture Notes for ECE 2277 (Digital Logic Systems)
		\item Digital Design With An Introduction to the Verilog HDL - M. Mano, D. Ciletti
		\item Verilog Complete Tutorial - VLSI Point (\href{https://www.youtube.com/playlist?list=PL_3xKnVkfI2itQhCyfnamNYSCHd2KHi4k}{YouTube Link})
	\end{itemize}
	\newpage
}


% END_FOLD PREFACE

% BEGIN_FOLD COVERPAGE
\newcommand{\makecoverpage}[5]{
	
	%%%% LAYERS %%%%
	\thispagestyle{empty}
	\pgfdeclarelayer{bg}
	\pgfdeclarelayer{main}
	\pgfdeclarelayer{fg}
	\pgfsetlayers{bg, main, fg}
	
	%%% DRAWING %%%
	\begin{tikzpicture}[remember picture, overlay]
		
		\begin{pgfonlayer}{fg}
			\fill[greyblue] (current page.south west) rectangle (current page.north east);
		\end{pgfonlayer}
		
	\end{tikzpicture}
	
	\begin{tikzpicture}[remember picture, overlay]
		
		\begin{pgfonlayer}{fg}
			% title and subtitle
			\node[align=right, text=white, anchor=east] at ([xshift=10cm] current page.center)
			{\Huge\bfseries\fontsize{40}{40}\selectfont #5};
			\node[align=right, text=white, anchor=east] at ([xshift=10cm, yshift=2cm] current page.center)
			{\Huge\bfseries\fontsize{40}{40}\selectfont #1};
			\node[align=right, text=whitesmoke, anchor=east] at ([xshift=10cm,yshift=-1.5cm]current page.center) {\Large\item\fontsize{20}{20}\selectfont #2};
			
			% author and date 
			\node[align=right, anchor=south east] at ([xshift=-1cm, yshift=2cm]current page.south east) {\Large\color{whitesmoke}#3};
			\node[align=right,, anchor=south east] at ([xshift=-1cm, yshift=1cm]current page.south east) {\Large\color{whitesmoke}#4};
		\end{pgfonlayer}
		
	\end{tikzpicture}
	
	\begin{tikzpicture}[remember picture, overlay]
		\begin{pgfonlayer}{bg}
			% Define the number of sides and the radius
			\def\numSides{7}
			\def\radius{3cm}
			
			% Loop to draw concentric polygons
			\foreach \i in {1,...,15}{
				\node[draw, grey, dash pattern= on 1pt off 5+2*\i pt, line width = 1 pt, inner sep = 1cm, regular polygon, regular polygon sides=\numSides, minimum size=2*\i*\radius, rotate= 15+5*\i, opacity=50] at (-3,-6) {};
			}
		\end{pgfonlayer}
	\end{tikzpicture}
	\newpage
}
% END_FOLD COVERPAGE

% BEGIN_FOLD ENVIRONMENTS

%----------------------------------------------------------------------------------------
%   HIGHLIGHT ENVIRONMENT
%----------------------------------------------------------------------------------------

\newtcolorbox{highlight}{
	colback=greyblue,
	colframe=whitesmoke,
	coltext=whitesmoke,
	boxrule=0.75pt,
	boxsep=4pt,
	arc=0pt,
	outer arc=0pt,
	enlarge bottom by=0.25cm,
	enlarge top by=0.15cm
}

%----------------------------------------------------------------------------------------
%   REMARK ENVIRONMENT
%----------------------------------------------------------------------------------------

\newtcolorbox{remark}{
	colback=greyblue,
	colframe=whitesmoke,
	coltext=whitesmoke,
	boxrule=0pt,
	leftrule=0.75pt,
	boxsep=4pt,
	arc=0pt,
	outer arc=0pt,
	enlarge bottom by=0.5cm,
}

%----------------------------------------------------------------------------------------
%   OUTLINE ENVIRONMENT
%----------------------------------------------------------------------------------------

\newtcolorbox{outline}[1][4pt]{
	enhanced,
	colback=greyblue,
	colframe=greyblue, % To essentially hide the frame, but we will draw the corners manually
	coltext=whitesmoke,
	boxrule=1pt,
	boxsep=#1,
	arc=1pt,
	outer arc=1pt,
	enlarge bottom by=0.5cm,
	overlay={
		% Top left corner
		\draw[whitesmoke,line width=1pt] 
		(frame.north west) -- ++(0,-0.25*\tcbtextheight)
		(frame.north west) -- ++(0.25*\tcbtextwidth,0);
		% Bottom right corner
		\draw[whitesmoke,line width=1pt]
		(frame.south east) -- ++(0,0.25*\tcbtextheight)
		(frame.south east) -- ++(-0.25*\tcbtextwidth,0);
	}
}

%----------------------------------------------------------------------------------------
%   LSTLISTING ENVIRONMENT
%----------------------------------------------------------------------------------------
\lstset
{ %Formatting for code in appendix
	language=Verilog,
	frame=single,
	basicstyle=\footnotesize\texttt,
	numbers=left,
	stepnumber=1,
	showstringspaces=false,
	tabsize=4,
	breaklines=true,
	breakatwhitespace=false,
	aboveskip=2em,
	belowcaptionskip=0.75em
}
\usepackage[font={color=white, it}]{caption}
\renewcommand{\lstlistingname}{\color{white} Snippet}
\lstset{ %
	basicstyle=\footnotesize\ttfamily,        % size of fonts used for the code
	breaklines=true,                 % automatic line breaking only at whitespace
	captionpos=b,                    % sets the caption-position to bottom
	commentstyle=\color{grey},    % comment style
	escapeinside={\%*}{*)},          % if you want to add LaTeX within your code
	keywordstyle=\color{blue!55!green},       % keyword style
	stringstyle=\color{green!75!white},     % string literal style
}

%----------------------------------------------------------------------------------------
%   DEFINITION ENVIRONMENT
%----------------------------------------------------------------------------------------
\newenvironment{definition}{
	\begin{itemize}[labelindent=5em,labelsep=0.25cm,leftmargin=*]
		\item[{{\itshape\scshape\color{purple!55!red}{Definition}}}]{}
	}
	{
	\end{itemize}
}

%----------------------------------------------------------------------------------------
%   THEORY ENVIRONMENT
%----------------------------------------------------------------------------------------
\newenvironment{theory}{
	\begin{itemize}[labelindent=4.2em,labelsep=0.5cm,leftmargin=*]
		\item[{{\itshape\scshape\color{red!55!purple}{Theorem}}}]
	}
	{
	\end{itemize}
}
%----------------------------------------------------------------------------------------
%   KEYWORD, CODE, EMPH & STROKE ENVIRONMENT
%----------------------------------------------------------------------------------------
\newcommand{\keyword}[1]{{{\color{green}{#1}\,}}}
\newcommand{\code}[1]{{\ttfamily\color{blue!55!green}\,{#1}\,}}
\newcommand{\str}[1]{{\ttfamily\color{green!55!white}\,{"#1"}\,}}
\renewcommand{\emph}[1]{} % renew emph style here as required
% END_FOLD ENVIRONMENTS

% BEGIN_FOLD PAGE NUMBS
\pagenumbering{gobble}
% END_FOLD PAGE NUMBS

% BEGIN_FOLD BLOCK DIAGRAMS
% Define block styles
\tikzstyle{block} = [rectangle, draw, minimum width=5em, text centered, minimum height=4em, fill=green!45!black]
\tikzstyle{input} = [coordinate] %[draw, ellipse, text centered, minimum height=2em, fill=greyblue!65!black]
\tikzstyle{output} = [coordinate] %[draw, ellipse, text centered, minimum height=2em, fill=greyblue!65!black]
\tikzstyle{line} = [draw, -latex']
\tikzstyle{arrow} = [draw, -latex']
\tikzset{
	inverter/.style={rectangle,draw,inner sep=2pt,minimum size=6mm},
	dot/.style={circle,inner sep=0pt,minimum size=0.5mm,draw,fill=black},
	buswidth/.style={decoration={
			markings,
			mark= at position 0.5 with {\node[font=\footnotesize] {/};\node[below=2pt] {\tiny #1};}
		}, postaction={decorate}}
}
% END_FOLD BLOCK DIAGRAMS

% BEGIN_FOLD CUSTOM CTIKZ
\tikzset{flipflop HA/.style={flipflop,
		flipflop def={t1=x, t3=y, t4=c, t6=r}}
}
% END_FOLD CUSTOM CTIKZ

% END_FOLD PREAMBLE

%

% BEGIN_FOLD DOCUMENT
\begin{document}
	
	\makecoverpage{Telecommunications}{Self Studies}{Arnav Goyal}{Winter 2024}{\& Satellite Operations}
	
	%
	
	\tableofcontents
	
	%
	
	\chapter{TELECOMMUNICATIONS}
	\section{BASICS}
	
	\keyword{Telecommunications} is the exchange of information through wire, radio, optical, or other EM based systems.
	
	Information can be transmitted wirelessly through EM-waves. Most commonly we use EM waves in the range of 300 kHz to 300 GHz (1km to 1mm wavelengths). 
	
	Radio waves for communication are created by a \keyword{transmitter} (abbreviated tx) and detected on the other side of a \keyword{radio link} by a \keyword{receiver} (abbreivated rx). A basic radio link is shown below.
	\\ % link diagram
	\begin{center} \begin{circuitikz}
			\draw (0,0) node[flipflop] (TX) {TX};
			\draw (TX.pin 5) to ++(-0.25,0) -| ++(2,2) node[bareTXantenna] {Tx};
			\draw (12,0) node[flipflop] (RX) {RX};
			\draw (RX.pin 2) to ++(0.25,0) -| ++(-2,2) node[bareRXantenna] {Rx};
	\end{circuitikz} \end{center}
	\vspace{1em}
	
	\section{NOISE}
	
	A receiver will receive everything in its reception bandwidth, including \keyword{noise}. Noise corrupts the signal and introduces errors into the received data.
	
	To help deal with this, an important figure of merit is called the \keyword{signal-to-noise ratio (SNR)}. It is defined as the ratio between the received signal power to the received noise power
	
	\begin{remark}
		\[ \text{SNR} = \frac{P_S}{P_N} \]
	\end{remark}
	
	\section{OTHER DECIBELS}
	
	Building off the base notion of \keyword{decibels}, where an increase of 10dB means a multiplication of 10. Recall that they are defined as:
	\begin{remark}
		\[G_{dB} = 10\log R\]
	\end{remark}
	Another useful quantity is \keyword{decibel-watts (dBW)}. To do this, we use a reference quantity, typically $P_0=1$W
	\begin{remark}
		\[P_{dBW} = 10\log \frac{P}{P_0}\]
	\end{remark}
	This lets us simplify gain-calculations using addition. A -27dbW signal going through a 26dB gain will have a resulting strength of -1dbW.
	%
	
	\section{ANTENNAS}
	
	An \keyword{antenna} is a device which couples EM-waves to and from free space. This means that they can be used to transmit and/or receive data. Each antenna also intrinsically has a \keyword{gain} which refers to the amount that they appear to amplify a transmitted or received signal.
	
	Antennas are generally considered to be \keyword{reciprocal}, which means that tx gain is the same as rx gain. Antennas also also \keyword{passive devices} so how does the concept of gain work?
	
	Gain is actually considered to be a geometric effect. It is essentially the focusing of a signal onto a point. This means that we actually need to point the antenna to the other end of the radio link!
	
	\chapter{MODULATION}
	A radio wave of constant frequency and amplitude carries no information - its just a wave. We can \keyword{encode information} into this wave by changing some of its properties in a way that both the TX and RX know, this process is referred to as \keyword{modulation}.
	
	Each EM wave has two properties we can adjust - its \keyword{amplitude} and \keyword{phase}. We can adjust these properties with respect to time to serially encode a message.
	
	\section{ON OFF KEYING}
	The simplest way to send a digital message is to turn the wave on and off to send a 1 or 0. This is called \keyword{on-off-keying} (OOK). In other words, we set amplitude $A=1$ to send a 1, and set $A=0$ to send a 0. We also leave the phase $\varphi$ untouched
	
	This is very easy to implement, but it is very susceptible to noise, and it features a wide range of spectral content.
	
	Due to these limitations OOK is only really used for simple and low-cost systems.
	
	\section{PHASE SHIFT KEYING}
	
	Here we do the opposite, we leave the amplitude $A$ untouched, and only change the phase $\varphi$.
	This is called \keyword{phase-shift-keying} (PSK). In binary PSK (BPSK) we choose phase shifts as far apart as possible (ex. 0, and $\pi$).
	
	This is much more robust to noise, and is moderately complex to implement. This is very commonly used for telemetry and control links between spacecraft and earth.
	
	\section{FREQUENCY SHIFT KEYING}
	
	We can also send a signal through \keyword{frequency modulation}. This is called \keyword{frequency-shift-keying} (FSK) and we can do this easily by choosing two frequencies, the lower one representing a 0, and the higher one representing a 1. This is called binary FSK (BFSK).
	
	This is popular due to its reasonable noise immunity, and it is more power efficient due to it using a continuous waveform. It is also reasonably easy to implement.
	
	\section{OTHER MODULATION TYPES}
	
	\keyword{Amplitude Phase Shift Keying} (APSK) is very similar to another form of modulation called Quadrature Amplitude Modulation (QAM), but designed to allow the use of more power-efficient amplifiers. This is common on space missions for very high rate data down-link.
	
	\keyword{Guassian Minimum Shift Keying} (GMSK) is a special-case of FSK. GMSK uses the minimum possible frequency shift (1/2 the baud rate) and a gaussian pulse shape to ensure spectral efficiency. This effectively combines the error performance of PSK with the power efficiency of FSK (only through use of a special algorithm on the Rx end). This is becoming more common on space missions.
	
	
\end{document}
% END_FOLD DOCUMENT