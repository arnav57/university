% BEGIN_FOLD PREAMBLE
\documentclass[]{report}

%BEGIN_FOLD IMPORTS
\usepackage[]{tikz}
\usetikzlibrary{shapes.geometric}
\usepackage[]{xcolor}
\usepackage[]{anyfontsize}
\usepackage[top=20mm, bottom=25mm, left=25mm, right=25mm]{geometry}
\usepackage[explicit]{titlesec}
\usepackage{fancyhdr}
\usepackage[most]{tcolorbox}
\usepackage{listings}
\usepackage[]{parskip}
\usepackage{caption}
\usepackage{enumitem}
\usepackage[hidelinks]{hyperref}
\hypersetup{linktoc=all}
\usepackage[outline]{contour}
\usepackage{soul}
\sethlcolor{hlcol}
% END_FOLD IMPORTS

% BEGIN_FOLD COLORS
\definecolor{covercol}{HTML}{171717}
\definecolor{maintextcol}{HTML}{CCCCCC}
\definecolor{kwcol}{HTML}{CCCCCC} %used for Keywords
\definecolor{defcol}{HTML}{FFFFFF} % used for Definition 98817B
\definecolor{thecol}{HTML}{98817B} % used for Theorems 98817B
\definecolor{hlcol}{HTML}{383838} % used for highlight
\pagecolor{covercol}
\AtBeginDocument{\color{maintextcol}}
% END_FOLD COLORS

% BEGIN_FOLD FONTS
\usepackage[defaultfam,tabular,lining]{montserrat} %% Option 'defaultfam'
%% only if the base font of the document is to be sans serif
\usepackage[T1]{fontenc}
\renewcommand*\oldstylenums[1]{{\fontfamily{Montserrat-TOsF}\selectfont #1}}
% END_FOLD FONTS

% BEGIN_FOLD TITLE FORMATS
\titleformat{\chapter}[hang]{\huge\bfseries\color{white}}{\thechapter . }{0pt}{\MakeUppercase{#1}}[\vspace{-1cm}]
\titleformat{\section}[hang]{\Large\bfseries\color{white}}{\thesection . }{0pt}{\MakeUppercase{#1}}
% END_FOLD TITLE FORMATS

% BEGIN_FOLD COVERPAGE
\newcommand{\makecoverpage}[5]{
	
	%%%% LAYERS %%%%
	\thispagestyle{empty}
	\pgfdeclarelayer{bg}
	\pgfdeclarelayer{main}
	\pgfdeclarelayer{fg}
	\pgfsetlayers{bg, main, fg}
	
	%%% COLORS %%%
	\definecolor{covercol}{HTML}{171717}
	\definecolor{subtextcol}{HTML}{999999}
	
	%%% DRAWING %%%
	\begin{tikzpicture}[remember picture, overlay]
		
		\begin{pgfonlayer}{fg}
			\fill[covercol] (current page.south west) rectangle (current page.north east);
		\end{pgfonlayer}
		
	\end{tikzpicture}
	
	\begin{tikzpicture}[remember picture, overlay]
		
		\begin{pgfonlayer}{fg}
			% title and subtitle
			\node[align=right, text=white, anchor=east] at ([xshift=10cm] current page.center)
			{\Huge\bfseries\fontsize{40}{40}\selectfont #5};
			\node[align=right, text=white, anchor=east] at ([xshift=10cm, yshift=2cm] current page.center)
			{\Huge\bfseries\fontsize{40}{40}\selectfont #1};
			\node[align=right, text=white, anchor=east] at ([xshift=10cm,yshift=-1.5cm]current page.center) {\Large\item\fontsize{20}{20}\selectfont #2};
			
			% author and date 
			\node[align=right, text=white, anchor=south east] at ([xshift=-1cm, yshift=2cm]current page.south east) {\Large\color{white}#3};
			\node[align=right, text=white, anchor=south east] at ([xshift=-1cm, yshift=1cm]current page.south east) {\Large\color{white}#4};
		\end{pgfonlayer}
		
	\end{tikzpicture}
	
	\begin{tikzpicture}[remember picture, overlay]
		\begin{pgfonlayer}{bg}
			% Define the number of sides and the radius
			\def\numSides{4}
			\def\radius{3cm}
			
			% Loop to draw concentric polygons
			\foreach \i in {1,...,15}{
				\node[draw, subtextcol, dash pattern= on 1pt off 5+2*\i pt, line width = 1 pt, inner sep = 1cm, regular polygon, regular polygon sides=\numSides, minimum size=2*\i*\radius, rotate= 15+5*\i, opacity=50] at (6,-6) {};
			}
		\end{pgfonlayer}
	\end{tikzpicture}
	\newpage
}
% END_FOLD COVERPAGE

% BEGIN_FOLD ENVIRONMENTS

%----------------------------------------------------------------------------------------
%   HIGHLIGHT ENVIRONMENT
%----------------------------------------------------------------------------------------

\newtcolorbox{highlight}{
	colback=covercol,
	colframe=maintextcol,
	coltext=maintextcol,
	boxrule=1pt,
	boxsep=4pt,
	arc=0pt,
	outer arc=0pt,
	enlarge bottom by=0.25cm,
	enlarge top by=0.15cm
}

%----------------------------------------------------------------------------------------
%   REMARK ENVIRONMENT
%----------------------------------------------------------------------------------------

\newtcolorbox{remark}{
	colback=covercol,
	colframe=maintextcol,
	coltext=maintextcol,
	boxrule=0pt,
	leftrule=1pt,
	boxsep=4pt,
	arc=0pt,
	outer arc=0pt,
	enlarge bottom by=0.5cm,
}

%----------------------------------------------------------------------------------------
%   OUTLINE ENVIRONMENT
%----------------------------------------------------------------------------------------

\newtcolorbox{outline}[1][4pt]{
	enhanced,
	colback=covercol,
	colframe=covercol, % To essentially hide the frame, but we will draw the corners manually
	coltext=maintextcol,
	boxrule=0pt,
	boxsep=#1,
	arc=0pt,
	outer arc=0pt,
	enlarge bottom by=0.5cm,
	overlay={
		% Top left corner
		\draw[maintextcol,line width=2pt] 
		(frame.north west) -- ++(0,-0.25*\tcbtextheight)
		(frame.north west) -- ++(0.25*\tcbtextwidth,0);
		% Bottom right corner
		\draw[maintextcol,line width=2pt]
		(frame.south east) -- ++(0,0.25*\tcbtextheight)
		(frame.south east) -- ++(-0.25*\tcbtextwidth,0);
	}
}

%----------------------------------------------------------------------------------------
%   LSTLISTING ENVIRONMENT
%----------------------------------------------------------------------------------------
\lstset
{ %Formatting for code in appendix
	language=Python,
	basicstyle=\footnotesize,
	numbers=left,
	stepnumber=1,
	showstringspaces=false,
	tabsize=4,
	breaklines=true,
	breakatwhitespace=false,
}

%----------------------------------------------------------------------------------------
%   DEFINITION ENVIRONMENT
%----------------------------------------------------------------------------------------
\newenvironment{definition}{
	\begin{itemize}[labelindent=5em,labelsep=0.25cm,leftmargin=*]
		\item[{{\itshape\scshape\color{defcol}{Definition}}}]{}
	}
	{
	\end{itemize}
}

%----------------------------------------------------------------------------------------
%   THEORY ENVIRONMENT
%----------------------------------------------------------------------------------------
\newenvironment{theory}{
	\begin{itemize}[labelindent=4.2em,labelsep=0.5cm,leftmargin=*]
		\item[{{\itshape\scshape\color{thecol}{Theorem}}}]
	}
	{
	\end{itemize}
}
%----------------------------------------------------------------------------------------
%   KEYWORD & STROKE ENVIRONMENT
%----------------------------------------------------------------------------------------
\newcommand{\keyword}[1]{{{\color{kwcol}{\hl{#1}}\,}}}
\newcommand{\stroke}[1]{{\contour{white}{\bfseries\color{black}{#1}}}}
% END_FOLD ENVIRONMENTS

% BEGIN_FOLD PAGE NUMBS
\pagenumbering{gobble}
% END_FOLD PAGE NUMBS

% END_FOLD PREAMBLE

% BEGIN_FOLD DOCUMENT
\begin{document}	
	
	
\makecoverpage{Microprocessors}{ECE 3375}{Arnav Goyal}{Winter 2023}{\& Microcomputers}
\tableofcontents

\chapter{INTRODUCTION \& REVIEW}
This chapter goes over a bit of the basic knowledge that we will build off-of for the rest of this document, almost everything here was taught in ECE-2277 (Digital Logic Systems).
\section{NUMBER SYSTEMS}


In this course we will most often be using \keyword{binary} (base-2) and \keyword{hexadecimal} (base-16) number systems. We also assign some predefined words to binary numbers of differing lengths.

\begin{highlight}
	\begin{definition}
		A \keyword{bit} is a binary number of length 1 - in other words it is a binary digit.
	\end{definition}
	\vspace{10pt}
	\begin{definition}
		A \keyword{nibble} is a 4-digit binary number.
	\end{definition}
	\vspace{10pt}
	\begin{definition}
		A \keyword{byte} is an 8-digit binary number.
	\end{definition}
	\vspace{10pt}
	\begin{definition}
		A \keyword{word} is a larger binary number - most commonly a 32-bit binary number
	\end{definition}
	\vspace{10pt}
	\begin{definition}
		A \keyword{double-word} is an even larger binary number - in this case a 64-bit binary number
	\end{definition}
\end{highlight}

Converting from binary to decimal is cumbersome and rarely-useful, instead converting decimal to hexadecimal going nibble-by-nibble is much more useful as memory addresses and other things are usually stored as hexadecimal numbers.

\begin{remark}
	The number $(1101\,1010)_2$ is equal to $(\text{DA})_{16}$
\end{remark}
	

\section{BINARY CODED DECIMAL}

\keyword{Binary Coded Decimal (BCD)} uses each nibble to represent a single decimal (base-10) digit. Because each nibble can store 16 different values and decimal only uses 10 values, using BCD is an inefficient way to store data.

\begin{remark}
	The number $(1653)_{10}$ can be written in BCD as $0001\,\,0110\,\,0101\,\,0011$
\end{remark}

There is no given way to tell if a given set of nibbles is a BCD number or a standard binary value. This is true for most numbers in this course. BCD is only really useful for converting binary to decimal - in case we need to hand-check computer outputs for example.

\section{NEGATIVE NUMBERS}
In a pure binary system, we can only use the symbols 0 and 1 for representation. In order to cope with the fact of not being able to use a negative symbol there are three ways ways to do this. The only one we will be using in this course is the \keyword{Two's Complement} as it is the one most practically used.

\begin{enumerate}[]
	\item Signed Magnitude
	\item One's Complement
	\item Two's Complement
\end{enumerate}

\begin{highlight}
	\begin{definition}
		The \keyword{Two's Complement} of a binary number is found by flipping the bits and adding one. We define a number as negative (according to this convention) if the \keyword{MSb} is 1
	\end{definition}
	\vspace{0.5cm}
	For example,
	$5_{10}$ is $(0000\,\,0101)_2$, thus $(-5)_{10}$ is $(1111\,\,1011)_2$ \\ \\
	The number $(1101\,\,1001)_2$ is negative, thus we flip bits and add one to find its magnitude. We get a two's complement of $(0010\,\,0111)_2$ which tells us the original binary number represents a $(-39)_{10}$
\end{highlight}

Given some random binary sequence such as: $1001\,\,0110$ What is the number?

It could be an unsigned decimal number of 150, a twos complement decimal number of -106, or even a BCD representation of 96. An important thing to remember is that \emph{there is no way to tell} without some additional information. These all look the same to the computer, and we impose meaning on the data by what we do with it.

\section{BINARY ARITHMETIC}
Binary addition is pretty easy, so I wont be going over it. The binary subtraction $A-B$ is performed by adding $A$ to the two's complement of $B$, which is then just an easy addition.

\begin{highlight}
	There is a major problem with binary arithmetic because binary numbers usually have a fixed (and finite) \keyword{width} \\
	
	\begin{definition}
		The \keyword{width} of a binary number is the amount of bits available to store that number. \\
	\end{definition}
	\begin{definition}
		\keyword{Overflow} occurs when the result of an arithmetic operation exceeds the available width for it to be stored.
	\end{definition}
\end{highlight}

There are two "kinds" of flags - \keyword{overflow} and \keyword{carry}, depending on if the operation is signed or unsigned. Essentially overflow represents a violation of width during signed addition, and carry represents a violation of width during unsigned addition.

When unsigned addition exceeds the available width it produces an extra bit called the \keyword{carry bit}, in computer systems this carry bit is essentially lost due it being larger than the available width, thus it returns a wrong result but enables the \keyword{carry flag}.

\begin{remark}
	Consider the following unsigned addition:
	\[	(11100)_2 + (00110)_2 = (1\,\,00010)_2	\]
	This addition produces an extra bit and violates the width of 5 bits, thus there is a carry generated
\end{remark}

During signed addition, the range of values we can store is basically halved to make space for negative numbers, Thus sometimes when adding signed numbers we exceed this range, which results in the enabling of the \keyword{overflow flag}

\begin{remark}
	Consider the following signed addition:
	\[	(01100)_2 + (00101)_2 = (10001)_2	\]
	This addition incorrectly changes the MSb from 0 to 1. If we interpret the operands as signed numbers according to the two's complement, the result must also be interpreted this way, and it results in the incorrect statement of $12+5=-15$.\\
	
	The real result of this operation is $12+5=17$, which lies outside the representable range of 5-bit two's complements, Hence the overflow.
\end{remark}

Some quick takeaways:
\begin{itemize}
	\item Overflow only corresponds with signed arithmetic
	\item Carry only corresponds with unsigned arithmetic
	\item An overflow can never occur when adding two numbers with differing signs
	\item During subtraction, carry = NOT borrow
\end{itemize}


\end{document}
% END_FOLD DOCUMENT
